\documentclass[a4paper]{report}
\usepackage{amssymb, amsfonts, amsmath, eurosym}
\usepackage{graphicx, import, wrapfig}
\usepackage{fixltx2e}
\usepackage[T1]{fontenc}
\usepackage[latin1]{inputenc}
\usepackage[italian]{babel}
\usepackage{vmargin}
\usepackage[usenames,dvipsnames]{color}
\usepackage[usenames,dvipsnames,svgnames,table]{xcolor}
\usepackage[italian]{varioref}
\usepackage{array}
%\usepackage{booktabs}
\usepackage{tikz,pgfplots,fp,ifthen}
\usepgfplotslibrary{fillbetween}
\usetikzlibrary{shapes,arrows,intersections, patterns}


\usepackage{hyperref}

\renewcommand*\arraystretch{1.5}

\newcommand{\OL}[1]{\overline{#1}}
\newcommand{\st}{\mathrm{s.t.}}
\newcommand{\Sc}[1]{\multicolumn{1}{c|}{#1}}
\newcommand{\Hr}[1]{%
  \colorbox{red!50}{$\displaystyle#1$}}
\newcommand*\C[1]{\tikz[baseline=(char.base)]{
    \node[shape=circle,draw,inner sep=2pt] (char) {#1};}}
    
\newcommand{\intne}[4]{\node [label={[above right=-4pt]45:#3}, name intersections={of=#1 and #2, by=#4}] at (#4) {$\bullet$}}
\newcommand{\intse}[4]{\node [label={[below right=-4pt]-45:#3}, name intersections={of=#1 and #2, by=#4}] at (#4) {$\bullet$}}
\newcommand{\intnw}[4]{\node [label={[above left=-4pt]135:#3}, name intersections={of=#1 and #2, by=#4}] at (#4) {$\bullet$}}
\newcommand{\intsw}[4]{\node [label={[below left=-4pt]-135:#3}, name intersections={of=#1 and #2, by=#4}] at (#4) {$\bullet$}}
\newcommand{\intn}[4]{\node [label={[above=-4pt]90:#3}, name intersections={of=#1 and #2, by=#4}] at (#4) {$\bullet$}}
\newcommand{\ints}[4]{\node [label={[below=-4pt]-90:#3}, name intersections={of=#1 and #2, by=#4}] at (#4) {$\bullet$}}
\newcommand{\inte}[4]{\node [label={[right=-4pt]0:#3}, name intersections={of=#1 and #2, by=#4}] at (#4) {$\bullet$}}
\newcommand{\intw}[4]{\node [label={[left=-4pt]180:#3}, name intersections={of=#1 and #2, by=#4}] at (#4) {$\bullet$}}

\newcommand{\dotgrid}[5]{%
    \foreach \x in {#1,...,#2} {%
    \foreach \y in {#3,...,#4}{%
    \edef\temp{%
    \noexpand\node [#5, draw, circle, inner sep=1pt] at (axis cs:\x,\y) {};%
    }\temp}%
    }}

%\newcolumntype{C}{>{\centering\arraybackslash$}p{\linewidth}<{$}}

\newcommand{\CG}[1]{\color{green}{#1}}

\includeonly{capitolo3}

\begin{document}
\setpapersize{A4}
\title{Esercizi di Ricerca Operativa}
\author{Simone Laierno}

\maketitle
\tableofcontents
\chapter*{Introduzione}
Questa �, o almeno si propone di essere, una raccolta degli esercizi proposti a lezione del corso Ricerca Operativa M tenuto dal prof. Silvano Martello all'interno del CdL di Ingegneria Informatica M dell'Universit� di Bologna.

Non ha pretese di esattezza, tutt'altro, ma spero sia d'aiuto a chi segue o seguir� il corso. \textbf{Qualsiasi errore, dubbio, correzione, ecc.} � pi� che bene accetto e pu� essere comunicato privatamente al mio contatto Facebook o al mio indirizzo \textbf{e-mail}: \href{mailto:simonelaierno@gmail.com}{simonelaierno@gmail.com}

\chapter{18/03/2014}

\section{Esercizio 1}

Sia dato - in linguaggio naturale - il seguente problema di ottimizzazione:
\begin{enumerate}
\item Un'azienda realizza due tipi di prodotti X e Y; 
\item Il profitto di 1T di prodotto Y � doppio di quello di 1T di prodotto X; 
\item La produzione di 1T di qualsiasi prodotto richiede 2 ore; 
\item Non si pu� comunque produrre per pi� di 9 ore;
\item Non si possono produrre pi� di 2T di Y; 
\item La produzione di X non pu� superare di pi� di una tonnellata la produzione di Y;
\end{enumerate}
Si modelli il problema come un problema di programmazione lineare, lo si porti in forma standard, si realizzi una rappresentazione grafica del problema e si ottimizzi la funzione di profitto attraverso il \textbf{metodo del simplesso} affinch� \textbf{si ottenga il massimo profitto dalla produzione dei prodotti} nel rispetto dei vincoli assegnati. Si utilizzi la \textit{regola di Dantzig} per scegliere le basi su cui fare pivot.

\subsection{Modellizzazione}
Si indichi con:
\begin{itemize}
\item $x_1$ il numero di tonnellate di prodotto X;
\item $x_2$ il numero di tonnellate di prodotto Y.
\end{itemize}
Lo scopo del nostro problema � di massimizzare i profitti ottenuti dalla produzione. Anche se non siamo a conoscenza degli esatti profitti dati da ogni prodotto, abbiamo comunque a disposizione la relazione data dalla proposizione 2, cio� la variabile $x_2$ rende il doppio della variabile $x_1$. Possiamo quindi esprimere cos� la funzione di profitto:
$$
\max z = x_1 + 2x_2
$$

Modelliamo ora i vincoli espressi dal problema.

Le relazione 3 e 4 ci impongono di non produrre per pi� di 9 ore, considerando che ogni tonnellata di prodotto richiede 2 ore per essere prodotta. Perci� il vincolo sar� espresso come:
$$
2x_1 + 2x_2 \leq 9
$$
La relazione 5 � molto semplice, indica semplicemente che non potremo produrre pi� di due tonnellate di prodotto Y:
$$
x_2 \leq 2
$$
L'ultima relazione (6) ci impone un limite superiore alla produzione del prodotto X, che non deve superare di pi� di una tonnellata la produzione del prodotto Y. Perci�:
$$
x_1 \leq x_2 + 1
$$
Infine imponiamo il vincolo, implicito, che la produzione non pu� essere negativa:
$$
x_1,x_2 \geq0
$$
Il modello matematico pu� essere quindi cos� riassunto:
\begin{align*}
\max z	&= x_1+2x_2 \\
\st\;\;2& x_1+2x_2 \leq 9\\
	  	& x_2 \leq 2\\
	  	& x_1 \leq x_2 + 1\\
	  	& x_1,x_2 \geq 0
\end{align*}

\subsection{Problema in forma grafica}

In figura \vref{fig:graph1} � rappresentato graficamente il problema presentato. In giallo � rappresentato il politopo $P$ e sono stati chiamati $\alpha,\beta,\gamma,\delta,\varepsilon$ i suoi cinque vertici, i quali sappiamo corrispondere ognuno ad una BFS.
Inoltre in figura � riportato il verso del gradiente della funzione obiettivo. Ricordiamo che � necessario che il politopo $P$ sia limitato nella direzione del gradiente o, pi� precisamente, che sia limitato nella direzione opposta al gradiente dopo aver trasformato la funzione obiettivo in una funzione di minimo. Ovviamente i due casi sono gli stessi, basti osservare che il problema � espresso equivalentemente dalle equazioni:
\begin{align*}
\max z&=x_1+2x_2 \\
\min \varphi=-z&=-x_1-2x_2
\end{align*}
I gradienti delle due funzioni $z$ e $\varphi$ sono perci�:
\begin{align*}
\nabla(z)&=\left(\frac{\partial z}{\partial x},\frac{\partial z}{\partial y}\right) = \left(1,2\right) \\
\nabla(\varphi)&=\left(\frac{\partial\varphi}{\partial x},\frac{\partial\varphi}{\partial y}\right) = -\nabla(z) = \left(-1,-2\right)
\end{align*}
I due vettori sono ovviamente uno l'opposto dell'altro e di conseguenza il gradiente di $z$ cresce dove decresce quello di $\varphi$. I problemi sono quindi equivalenti.

\begin{figure}[htbp]
\centering
\begin{tikzpicture}
\begin{axis}
[axis lines=middle, axis equal, enlargelimits, xlabel=$x_1$, ylabel=$x_2$,
 every axis x label/.style={
    at={(ticklabel* cs:1.01)},
    anchor=west,
 },
 every axis y label/.style={
    at={(ticklabel* cs:1.01)},
    anchor=south,
 },]
    \path[name path=AX] 
        (axis cs:\pgfkeysvalueof{/pgfplots/xmin},0)--
        (axis cs:\pgfkeysvalueof{/pgfplots/xmax},0);
    \path[name path=AY] 
        (axis cs:0,\pgfkeysvalueof{/pgfplots/ymin})--
        (axis cs:0,\pgfkeysvalueof{/pgfplots/ymax});
    \path[name path=UP]
    	(axis cs:\pgfkeysvalueof{/pgfplots/xmin},\pgfkeysvalueof{/pgfplots/ymax})--
    	(axis cs:\pgfkeysvalueof{/pgfplots/xmax},\pgfkeysvalueof{/pgfplots/ymax});
\addplot
[domain=0:4.5, samples=10, thick, blue, name path=2x2y9]
{-x+(9/2)} node [pos=0.15,pin={75:{\color{blue}$2x_1+2x_2=9$}}, inner sep=0pt] {};
\addplot
[domain=1:5, samples=10, thick, red, name path=yx-1]
{x-1} node [pos=0.7, anchor=east, pin={165:{\color{red}$x_1=x_2+1$}}, inner sep=0pt] {};
\addplot
[domain=0:5, samples = 10, thick, purple, name path=y2]
{2} node [pos=0.1, anchor=north, pin={90:{\color{purple}$x_2=2$}}, inner sep= 0pt] {};
\addplot[thick, fill=yellow, fill opacity=0.5] fill between [of=2x2y9 and AX, soft clip={domain=0:11/4},];
\addplot[white] fill between [of=2x2y9 and y2];
\addplot[pattern=north west lines, pattern color=red!10] fill between [reverse=true, of=AX and UP, soft clip={domain=0:5}];
\addplot[white] fill between [of=yx-1 and AX];
\addplot[pattern=north east lines, pattern color=blue!10] fill between [of=AX and 2x2y9, soft clip={domain=0:5}];
\addplot[pattern=vertical lines, pattern color=purple!10] fill between [of=AX and y2];
%\node [label={[above right=-6pt]45:$\alpha$}, name intersections={of=AX and AY, by=alp}] at (alp) {$\bullet$};
%\node [label={[below right=-6pt]-45:$\beta$}, name intersections={of=AY and y2, by=bet}] at (bet) {$\bullet$};
%\node [label={[above right=-6pt]45:$\gamma$}, name intersections={of=y2 and 2x2y9, by=gam}] at (gam) {$\bullet$};
%\node [name intersections={of=2x2y9 and yx-1, by=del}, label={[right=-4pt]0:$\delta$}] at (del) {$\bullet$};
%\node [name intersections={of=yx-1 and AX, by=eps}, label={[above=-6pt]90:$\epsilon$}] at (eps) {$\bullet$};
\intse{AX}{AY}{$\alpha$}{alp};
\intse{AY}{y2}{$\beta$}{bet};
\intne{y2}{2x2y9}{$\gamma$}{gam};
\inte{2x2y9}{yx-1}{$\delta$}{del};
\intn{yx-1}{AX}{$\varepsilon$}{eps};
\node at (axis cs:1,1) {$P$};
\addplot[-latex, thick] coordinates
           {(0,0) (1/2.24,2/2.24)} node [pos=1, anchor=north, label={90:{\small $\nabla z$}}] {};
\end{axis}
\end{tikzpicture}
\caption{Rappresentazione cartesiana del problema di programmazione lineare}
\label{fig:graph1}
\end{figure}

\subsection{Forma standard}
Ricordiamo che un problema di \textbf{programmazione lineare in forma standard} � nella forma (matriciale):
\begin{align*}
\min c'x& \\
Ax& = b \\
x& \geq 0
\end{align*}
Cio�, la funzione obiettivo deve essere sotto forma di minimo (il che � molto semplice, dato che basta moltiplicarla per $-1$), i vincoli devono essere tutti espressi sotto forma di \textit{equazioni} e tutte le variabili devono essere positive.
A tal scopo introduciamo una \textbf{variabile slack} per ogni disequazione con simbolo $\leq$.
\begin{alignat*}{7}
&\min \varphi = \quad && -x_1 \quad && -2x_2\\
&\;\st  &&+2x_1		&&+2x_2 		&&+\pmb{x_3}	&&		 		&&\qquad\qquad		&&=9\\
&	 	&&\qquad\qquad &&+x_2		&&\qquad\qquad	&& +\pmb{x_4}	&&					&&=2\\
&	 	&&+x_1		&&-x_2 \qquad	&&				&&\qquad\qquad	&&+\pmb{x_5}		&&=1\\
&		&&\quad\; x_1,	&&\quad\; x_2,		&&\quad\; x_3,		&&\quad\; x_4,		&&\quad\; x_5		&&\geq 0
\end{alignat*}

\subsection{Risoluzione tramite tableau}

\subsubsection{Richiami (molto blandi) di teoria}
Una \textbf{base} � determinata da una sottomatrice della matrice A dei vincoli, di dimensione $m\times n$, di $n$ colonne linearmente indipendenti (nel \textbf{tableau} la matrice A � quella delimitata dalle due righe disegnate). Spesso l'individuazione delle colonne della prima base � semplice perch� l'introduzione di variabili slack o di variabili surplus crea nella nostra matrice delle colonne con tutti 0 e solo un 1, il che rende probabile la formazione di una sottomatrice \textbf{identit�}.
Ricordiamo che scelta una base $\mathcal{B}$ tale che:
$$
\mathcal{B}=A_{\beta(i)}; \quad i=1,\ldots,m
$$
ad essa � associata una \textbf{soluzione base} $x$ tale che:
$$
x=x_j; \quad j=1,\ldots,n \\
x_j = 0 \quad \forall j : A_j\not\in \mathcal{B}
$$
cio� il valore di una variabile non in base � 0. Questa � inoltre detta una soluzione \textbf{ammissibile} (\textbf{BFS}) se si trova nelle regione ammissibile determinata dai vincoli.

Al tableau aggiungeremo in alto una riga che indicher� il \textbf{costo relativo} $\OL{c_j}$ della colonna $j$-esima. Basti sapere che se facciamo in modo che $\forall A_j \in \mathcal{B} : \OL{c_j} = 0$, avremo nella prima colonna il guadagno $-\varphi$ della funzione obiettivo e in tutti gli altri avremo effettivamente i costi relativi. Per una spiegazione del perch� di questo fenomeno magico, si rimanda al testo o alle slide del docente.

Si ricorda, infine, che la colonna $b$ dei termini noti verr� inserita a sinistra. Non � indispensabile, ma una semplice convenzione.

\subsubsection{Risoluzione}

Per realizzare il \textbf{tableau} � sufficiente ricordare le regole base. La matrice $A$ e la colonna $b$ si riportano fedelmente sotto la loro consueta forma di matrice. Le variabili $x_j$ sono i coefficienti della rispettiva variabile nella funzione obiettivo. Ovviamente per tutte le variabili slack e surplus, che sono state aggiunte artificialmente da noi, il loro valore � 0. 

La prima colonna, una volta scelta una base che ha la forma di una matrice identit� (e per ora assumeremo che sia sempre gi� pronta o facilmente costruibile) rappresenta banalmente la soluzione del sistema $Ix=b$ che altro non � che $b$ stesso.

L'ultimo valore da inserire � quello di $-\varphi$, che varie a seconda delle colonne che assumeremo inizialmente come base. Se, come spesso accadr�, scegliamo tutte colonne associate a variabili slack o surplus, il loro valore non influir� sulla funzione obiettivo ed essendo tutte le altre variabili automaticamente nulle perch� non sono in base, sar� nullo anche $-\varphi$. Il nostro caso attuale ricade in quest'ultimo descritto, ma se fosse stato altrimenti, avremmo semplicemente dovuto calcolare il valore di $-\varphi$ in base al valore delle variabili $x_1$ e $x_2$.

\begin{table}[htbp]
\centering
\begin{tabular}{rcccccc}
 & $-\varphi$ & $x_1$ & $x_2$ & $x_3$ & $x_4$ & $x_5$ \\
$\OL{c_j}$ & \Sc{0} & -1 & -2 & 0 & 0 & 0 \\
\cline{2-7}
$x_3$ & \Sc{9} & 2 & 2 & 1 & 0 & 0 \\
$x_4$ & \Sc{2} & 0 & 1 & 0 & 1 & 0 \\
$x_5$ & \Sc{1} & 1 & -1 & 0 & 0 & 1 \\
\end{tabular}
\caption{Tableau iniziale. Vertice $\alpha(0,0)$}
\label{tab:tab1}
\end{table}

In tabella \vref{tab:tab1} il tableau definitivo ricavato dal nostro problema. Si noti che le ultime 3 colonne formano gi� una matrice identit�, perci� le assumeremo come base.
\begin{align*}
\mathcal{B}&=\{A_3,A_4,A_5\}\\
x&=(0,0,9,2,1)
\end{align*}

Per sapere in che punto dello spazio originale a due dimensioni ci troviamo, � sufficiente guardare le variabili $x_1$ e $x_2$. \'E evidente che ci troviamo nell'origine, che appartiene al politopo $P$ trovato in precedenza e che in particolare � il \textbf{vertice} $\pmb{\alpha}$.
Poich� non tutti i $\OL{c_j}$ sono non negativi, la nostra non � la BFS ottima e dobbiamo muoverci in una BFS migliore. Applicando la \textbf{regola di Dantzig}, facciamo entrare in base la colonna con il costo relativo maggiore in valore assoluto (cio� il "pi� negativo"). Nel nostro caso, prenderemo in considerazione quindi la colonna $\pmb{A_2}$.
Per scegliere su quale elemento fare \textbf{pivoting}, dobbiamo ottenere il valore di $y_{\ell 2}$ tale che:
$$
\vartheta_{\max}=\min_{i:y_{i2}>0}\frac{y_{i0}}{y_{i2}}=\frac{y_{i0}}{y_{\ell 2}}
$$
Perci�, operando con gli elementi nel tableau:
\begin{align*}
\vartheta_{\max}=\min\left(\frac{9}{2},\frac{2}{1}\right)=\frac{2}{1}=\frac{y_{20}}{\pmb{y_{22}}}
\end{align*}
Faremo pivoting sull'elemento $y_{22}$ (cerchiato in tabella \vref{tab:tab2}). Il nostro scopo � ora far comparire uno 0 nella colonna dell'elemento pivot in tutte le righe tranne quella in cui si trova l'elemento pivot e far comparire un 1 in quest'ultima.

\begin{table}[htbp]
\centering
\begin{tabular}{rrcccccc}
 & & $-\varphi$ & $x_1$ & $x_2$ & $x_3$ & $x_4$ & $x_5$ \\
$R_0$ & $\OL{c_j}$ & \Sc{0} & -1 & -2 & 0 & 0 & 0 \\
\cline{3-8}
$R_1$ & $x_3$ & \Sc{9} & 2 & 2 & 1 & 0 & 0 \\
$R_2$ & $x_4$ & \Sc{2} & 0 & \C{1} & 0 & 1 & 0 \\
$R_3$ & $x_5$ & \Sc{1} & 1 & -1 & 0 & 0 & 1 \\
\end{tabular}
\caption{Pivoting su $y_{22}$. $A_2$ entra in base e $A_4$ esce.}
\label{tab:tab2}
\end{table}

Possiamo felicemente notare che $y_{22}=1$, perci� nulla da fare su $R_2$. Se cos� non fosse stato sarebbe bastato moltiplicare $R_2R$ per un coefficiente $h$. Algebricamente, per far comparire uno 0 in tutti gli altri elementi della colonna $A_2$, possiamo sostituire ad ogni riga la riga stessa sommata ad un'altra qualsiasi riga moltiplicata per un coefficiente $k$. Ovviamente la riga pi� comoda da sommare � la riga su cui stiamo facendo pivot $R_\ell$, avendo un comodissimo 1 nella colonna interessata. Perci� possiamo riassumere che l'operazione consentita su ogni riga $R_i$ e sulla riga di pivot $R_\ell$�:
\begin{align*}
R_\ell&\leftarrow hR_l \\
R_i&\leftarrow R_i+kR_\ell
\end{align*}
Queste sono dette \textbf{operazioni elementari di riga}. Applicando le regole al nostra tableau, operiamo:
\begin{align*}
R_0&\leftarrow R_0 + 2R_2; \\
R_1&\leftarrow R_1 - 2R_2; \\
R_3&\leftarrow R_3 + R_2
\end{align*}
Il nostro nuovo tableau diventa quindi quello in tabella \vref{tab:tab3}.

\begin{table}[htbp]
\centering
\begin{tabular}{rrcccccc}
 & & $-\varphi$ & $x_1$ & $x_2$ & $x_3$ & $x_4$ & $x_5$ \\
$R_0$ & $\OL{c_j}$ & \Sc{4} & -1 & 0 & 0 & 2 & 0 \\
\cline{3-8}
$R_1$ & $x_3$ & \Sc{5} & 2 & 0 & 1 & -2 & 0 \\
$R_2$ & $x_2$ & \Sc{2} & 0 & 1 & 0 & 1 & 0 \\
$R_3$ & $x_5$ & \Sc{3} & 1 & 0 & 0 & 1 & 1 \\
\end{tabular}
\caption{Secondo tableau. Vertice $\beta(0,2)$}
\label{tab:tab3}
\end{table}

Ora che $A_4$ � entrato in base e $A_2$ ne � uscito, abbiamo una nuova base $\mathcal{B}$ e una nuova BFS $x$:
\begin{align*}
\mathcal{B}&=\{A_3,A_2,A_5\} \\
x&=(0,2,5,0,3)
\end{align*}
Ci troviamo nel \textbf{vertice} $\pmb{\beta}$, ma questa non � ancora la BFS ottima. Possiamo osservare, infatti, che la colonna $A_1$ presenta un $\OL{c_j}$ negativo e sar� necessario fare ulteriormente pivoting su un elemento di questa. Otteniamo quindi il valore di $y_{\ell 1}$ tale che:
\begin{align*}
\vartheta_{\max}&=\min_{i:y_{i1}>0}\frac{y_{i0}}{y_{i1}}=\frac{y_{i0}}{y_{\ell 1}} \\
\vartheta_{\max}&=\min\left(\frac{5}{2},\frac{3}{1}\right)=\frac{5}{2}=\frac{y_{10}}{\pmb{y_{11}}}
\end{align*}
Faremo pivoting sull'elemento $y_{11}$ (cerchiato in tabella \vref{tab:tab4}). 
\begin{table}[htbp]
\centering
\begin{tabular}{rrcccccc}
 & & $-\varphi$ & $x_1$ & $x_2$ & $x_3$ & $x_4$ & $x_5$ \\
$R_0$ & $\OL{c_j}$ & \Sc{4} & -1 & 0 & 0 & 2 & 0 \\
\cline{3-8}
$R_1$ & $x_3$ & \Sc{5} & \C{2} & 0 & 1 & -2 & 0 \\
$R_2$ & $x_2$ & \Sc{2} & 0 & 1 & 0 & 1 & 0 \\
$R_3$ & $x_5$ & \Sc{3} & 1 & 0 & 0 & 1 & 1 \\
\end{tabular}
\caption{Pivoting su $y_{11}$. $A_1$ entra in base e $A_3$ esce.}
\label{tab:tab4}
\end{table}

Questa volta dobbiamo lavorare anche sulla riga dell'elemento pivot, dividendola per 2:
$$
R_1\rightarrow \frac{1}{2}R_1
$$
Partendo dal nuovo tableau in tabella \vref{tab:tab5}, facciamo pivoting sulle restanti righe in questo modo:
\begin{align*}
R_0&\rightarrow R_0 + R_1 \\
R_3&\rightarrow R_3 - R_1
\end{align*}
\begin{table}[htbp]
\centering
\begin{tabular}{rrcccccc}
 & & $-\varphi$ & $x_1$ & $x_2$ & $x_3$ & $x_4$ & $x_5$ \\
$R_0$ & $\OL{c_j}$ & \Sc{4} & -1 & 0 & 0 & 2 & 0 \\
\cline{3-8}
$R_1$ & $x_1$ & \Sc{$\frac{5}{2}$} & \C{1} & 0 & $\frac{1}{2}$ & -1 & 0 \\
$R_2$ & $x_2$ & \Sc{2} & 0 & 1 & 0 & 1 & 0 \\
$R_3$ & $x_5$ & \Sc{3} & 1 & 0 & 0 & 1 & 1 \\
\end{tabular}
\caption{Pivoting su $y_{11}$. $R_1$ divisa per 2.}
\label{tab:tab5}
\end{table}
Il nostro nuovo tableau, quindi, � quello in tabella \vref{tab:tab6}. 
\begin{table}[htbp]
\centering
\begin{tabular}{rrcccccc}
 & & $-\varphi$ & $x_1$ & $x_2$ & $x_3$ & $x_4$ & $x_5$ \\
$R_0$ & $\OL{c_j}$ & \Sc{$\frac{13}{2}$} & 0 & 0 & $\frac{1}{2}$ & 1 & 0 \\
\cline{3-8}
%\hline
$R_1$ & $x_1$ & \Sc{$\frac{5}{2}$} & 1 & 0 & $\frac{1}{2}$ & -1 & 0 \\
$R_2$ & $x_2$ & \Sc{2} & 0 & 1 & 0 & 1 & 0 \\
$R_3$ & $x_5$ & \Sc{$\frac{1}{2}$} & 0 & 0 & $-\frac{1}{2}$ & 2 & 1 \\
\end{tabular}
\caption{Terzo tableau. Vertice $\gamma\left(\frac{5}{2},2\right)$}
\label{tab:tab6}
\end{table}
Notiamo che tutti i $\OL{c_j}$ sono non negativi, perci� ci troviamo nella \textbf{BFS ottima}. La base $\mathcal{B}$ e la soluzione $x$ sono quindi:
\begin{align*}
\mathcal{B}&={A_1,A_2,A_5} \\
x&=\left(\frac{5}{2},2,0,0,\frac{1}{2}\right)
\end{align*}
La soluzione ottima � quella del vertice $\pmb{\gamma\left(\frac{5}{2},2\right)}$. Riassumendo, tutti i valori delle variabili in gioco sono i seguenti:
\begin{align*}
z&=-\varphi=\frac{13}{2}=6.5 \\
x_1&=\frac{5}{2}=2.5 \\
x_2&=2
\end{align*}
\subsection{Conclusione}
La soluzione ottima consiste nel produrre $2.5$T di prodotto X e $2$T di prodotto Y, ottenendo un \textbf{profitto} pari a $6.5$ volte quello di $1$T di prodotto X.

\section{Esercizio 2}

Sia dato - in linguaggio naturale - il seguente problema di ottimizzazione:
\begin{enumerate}
\item Un'azienda chimica produce due composti, 1 e 2, composti da due sostanze chimiche A e B;
\item Un lotto di composto 1 richiede $3$T di sostanza A e $3$T di sostanza B;
\item Un lotto di composto 2 richiede $6$T di sostanza A e $3$T di sostanza B;
\item Per motivi di mercato non si possono produrre pi� di 3 lotti di composto 1;
\item Si hanno a disposizione $12$T di composto A e $9$T di composto B;
\item Il profitto di un lotto di composto A � di $12\,000$\euro ;
\item Il profitto di un lotto di composto B � di $15\,000$\euro .
\end{enumerate}
Si modelli il problema come un problema di programmazione lineare, lo si porti in forma standard, si realizzi una rappresentazione grafica del problema e si ottimizzi la funzione di profitto attraverso il \textbf{metodo del simplesso} affinch� \textbf{si ottenga il massimo profitto dalla produzione dei prodotti} nel rispetto dei vincoli assegnati. Si utilizzi la \textit{regola di Bland} per scegliere le basi su cui fare pivot.

\subsection{Modellizzazione}
Si indichi con:
\begin{itemize}
\item $x_1$ il numero di lotti di composto 1;
\item $x_2$ il numero di lotti di composto 2.
\end{itemize}
Lo scopo del nostro problema � di massimizzare i profitti ottenuti dalla produzione. Per comodit� di rappresentazione, stabiliamo che la funzione di profitto $z$ esprima il profitto in \textbf{migliaia di euro}:
$$
\max z = 12x_1 + 15x_2
$$

Modelliamo ora i vincoli espressi dal problema.

Dalle relazioni 3, 4 e 5 possiamo dedurre i seguenti vincoli:
\begin{itemize}
\item Servono $3$T di prodotto A per produrre un lotto di composto 1 e $6$T per produrre un lotto di composto 2. In tutto non possiamo utilizzare pi� di $12$T di prodotto A.
\item Servono $3$T di prodotto B per produrre un lotto di composto 1 e $3$T per produrre un lotto di composto 2. In tutto non possiamo utilizzare pi� di $9$T di prodotto B.
\end{itemize}
\begin{align*}
3x_1 + 6x_2 &\leq 12 \\
3x_1 + 3x_2 &\leq 9
\end{align*}
La relazione 4 � cos� facilmente esprimibile:
$$
x_1 \leq 3
$$
Infine imponiamo il vincolo, implicito, che la produzione non pu� essere negativa:
$$
x_1,x_2 \geq0
$$
Il modello matematico pu� essere quindi cos� riassunto (sono state apportate semplificazione algebriche):
\begin{align*}
\max z	&= 12x_1+15x_2 \\
\st\;\; & x_1+2x_2 \leq 4\\
	  	& x_1+x_2 \leq 3\\
	  	& x_1 \leq 3\\
	  	& x_1,x_2 \geq 0
\end{align*}

\subsection{Problema in forma grafica}

In figura \vref{fig:graph2} � rappresentato graficamente il problema presentato. In giallo � rappresentato il politopo $P$ e sono stati chiamati $\alpha,\beta,\gamma,\delta$ i suoi quattro vertici, i quali sappiamo corrispondere ognuno ad una BFS.
Il gradiente della funzione obiettivo vale
\begin{equation*}
\nabla(z)=\left(\frac{\partial z}{\partial x},\frac{\partial z}{\partial y}\right) = \left(12,15\right) \\
\end{equation*}
Il politopo $P$ �, ovviamente, limitato nella direzione del gradiente (si fa notare che finora $P$ � sempre limitato in ogni direzione, quindi qualsiasi direzione avesse il gradiente non ci sarebbero problemi).

\begin{figure}[htbp]
\centering
\begin{tikzpicture}
\begin{axis}
[axis lines=middle, axis equal, enlargelimits, xlabel=$x_1$, ylabel=$x_2$,
 every axis x label/.style={
    at={(ticklabel* cs:1.01)},
    anchor=west,
 },
 every axis y label/.style={
    at={(ticklabel* cs:1.01)},
    anchor=south,
 },]
    \path[name path=AX] 
        (axis cs:\pgfkeysvalueof{/pgfplots/xmin},0)--
        (axis cs:\pgfkeysvalueof{/pgfplots/xmax},0);
    \path[name path=AY] 
        (axis cs:0,\pgfkeysvalueof{/pgfplots/ymin})--
        (axis cs:0,\pgfkeysvalueof{/pgfplots/ymax});
    \path[name path=UP]
    	(axis cs:\pgfkeysvalueof{/pgfplots/xmin},\pgfkeysvalueof{/pgfplots/ymax})--
    	(axis cs:\pgfkeysvalueof{/pgfplots/xmax},\pgfkeysvalueof{/pgfplots/ymax});
\addplot
[domain=0:4, samples=10, thick, blue, name path=x2y4]
{-.5*x+2} node [pos=0.15,pin={75:{\color{blue}$x_1+2x_2=4$}}, inner sep=0pt] {};
\addplot
[domain=0:3, samples=10, thick, red, name path=xy3]
{-x+3} node [pos=0.1, pin={75:{\color{red}$x_1+x_2=3$}}, inner sep=0pt] {};
\addplot
[domain=0:5, samples = 10, thick, purple, name path=x3]
(3,x) node [pos=0.5, anchor=north, pin={0:{\color{purple}$x_1=3$}}, inner sep= 0pt] {};
\addplot[thick, fill=yellow, fill opacity=0.5] fill between [of=x2y4 and AX, soft clip={domain=0:3}];
\addplot[white] fill between [of=xy3 and UP];
%\addplot[pattern=north east lines, pattern color=red!10] fill between [reverse=true, of=AX and UP, soft clip={domain=0:5}];
%\addplot[white] fill between [of=xy3 and AX];
\addplot[pattern=north east lines, pattern color=red!10] fill between [of=xy3 and AX];
\addplot[pattern=vertical lines, pattern color=blue!10] fill between [of=x2y4 and AX];
%\addplot[pattern=north east lines, pattern color=blue!10] fill between [of=AX and 2x2y9, soft clip={domain=0:5}];
\addplot[pattern=horizontal lines, pattern color=purple!10] fill between [of=AY and x3];
\intse{AX}{AY}{$\alpha$}{alp};
\intne{AY}{x2y4}{$\beta$}{bet};
\intne{x2y4}{xy3}{$\gamma$}{gam};
\intne{xy3}{AX}{$\delta$}{del};
\node at (axis cs:1.5,.5) {$P$};
\addplot[-latex, thick] coordinates
           {(0,0) (4/6.4,5/6.4)} node [pos=1, anchor=north, label={90:{\small $\nabla z$}}] {};
\end{axis}
\end{tikzpicture}
\caption{Rappresentazione cartesiana del problema di programmazione lineare}
\label{fig:graph2}
\end{figure}

\subsection{Forma standard}
Ricordiamo che un problema di \textbf{programmazione lineare in forma standard} � nella forma (matriciale):
\begin{align*}
\min c'x& \\
Ax& = b \\
x& \geq 0
\end{align*}
Trasformiamo la funzione obiettivo $z$ in $\varphi$ tale che:
\begin{equation*}
\varphi=-\frac{z}{3}=-4x_1-5x_2
\end{equation*}
Quindi introduciamo una \textbf{variabile slack} per ogni disequazione con simbolo $\leq$. Otterremo infine:
\begin{alignat*}{7}
&\min \varphi = \quad && -4x_1 \quad && -5x_2\\
&\;\st  &&+x_1		&&+2x_2 		&&+\pmb{x_3}	&&		 		&&\qquad\qquad		&&=4\\
&	 	&&+x_1		&&+x_2			&&\qquad\qquad	&& +\pmb{x_4}	&&					&&=3\\
&	 	&&+x_1		&&\qquad\qquad	&&				&&\qquad\qquad	&&+\pmb{x_5}		&&=3\\
&		&&\quad\; x_1,	&&\quad\; x_2,		&&\quad\; x_3,		&&\quad\; x_4,		&&\quad\; x_5		&&\geq 0
\end{alignat*}

\subsection{Risoluzione tramite tableau}

\begin{table}[htbp]
\centering
\begin{tabular}{rcccccc}
			&$-\varphi$ & $x_1$ & $x_2$ & $x_3$ & $x_4$ & $x_5$ \\
$\OL{c_j}$ 	& \Sc{0} 	& -4 	& -5 	& 0 	& 0 	& 0 \\
\cline{2-7}
$x_3$ 		& \Sc{4} 	& 1 	& 2 	& 1 	& 0 	& 0 \\
$x_4$ 		& \Sc{3} 	& 1 	& 1		& 0 	& 1 	& 0 \\
$x_5$ 		& \Sc{3} 	& 1 	& 0 	& 0 	& 0 	& 1 \\
\end{tabular}
\caption{Tableau iniziale. Vertice $\alpha(0,0)$}
\label{tab:tab21}
\end{table}

In tabella \vref{tab:tab21} il tableau ricavato dal nostro problema. Si noti che le ultime 3 colonne formano gi� una matrice identit�, perci� le assumeremo come base.
\begin{align*}
\mathcal{B}&=\{A_3,A_4,A_5\}\\
x&=(0,0,4,3,3)
\end{align*}
Ci troviamo nell'origine, che appartiene al politopo $P$ trovato in precedenza e che in particolare � il \textbf{vertice} $\pmb{\alpha}$.
Poich� non tutti i $\OL{c_j}$ sono non negativi, la nostra non � la BFS ottima e dobbiamo muoverci in una BFS migliore. Applicando la \textbf{regola di Bland}, facciamo entrare in base la colonna con l'indice minore. Nel nostro caso, prenderemo in considerazione quindi la colonna $\pmb{A_1}$.
Per scegliere su quale elemento fare \textbf{pivoting}, dobbiamo ottenere il valore di $y_{\ell 1}$ tale che:
$$
\vartheta_{\max}=\min_{i:y_{i1}>0}\frac{y_{i0}}{y_{i1}}=\frac{y_{i0}}{y_{\ell 1}}
$$
Perci�, operando con gli elementi nel tableau:
\begin{align*}
\vartheta_{\max}=\min\left(\frac{4}{1},\frac{3}{1},\frac{3}{1}\right)=\frac{3}{1}=\frac{y_{20}}{\pmb{y_{21}}}=\frac{y_{30}}{\pmb{y_{31}}}
\end{align*}
Abbiamo un \textbf{pareggio} tra gli elementi $y_{21}$ e $y_{31}$.Seguendo la \textbf{regola di Bland}, sceglieremo come pivot l'elemento che far� uscire dalla base la variabile con l'\textbf{indice minore}%
\footnote{Si fa notare che l'utilizzo della \textbf{regola di Bland} evita i casi di \textbf{loop} in presenza di basi degeneri durante l'algoritmo del simplesso. Questa propriet� � dimostrabile ma la dimostrazione esula dai nostri scopi.}.
Faremo quindi pivoting sull'elemento $y_{21}$ (cerchiato in tabella \vref{tab:tab22}) poich� far� uscire dalla base la variabile $x_4$. Il nostro scopo � ora far comparire uno 0 nella colonna dell'elemento pivot in tutte le righe tranne quella in cui si trova l'elemento pivot e far comparire un 1 in quest'ultima.
\begin{table}[htbp]
\centering
\begin{tabular}{rrcccccc}
 	  & 			&$-\varphi$ & $x_1$ & $x_2$ & $x_3$ & $x_4$ & $x_5$ \\
$R_0$ & $\OL{c_j}$ 	& \Sc{0} 	& -4 	& -5 	& 0 	& 0 	& 0 \\
\cline{3-8}
$R_1$ & $x_3$ 		& \Sc{4} 	& 1 	& 2 	& 1 	& 0 	& 0 \\
$R_2$ & $x_4$ 		& \Sc{3} 	& \C{1}	& 1		& 0 	& 1 	& 0 \\
$R_3$ & $x_5$ 		& \Sc{3} 	& 1 	& 0 	& 0 	& 0 	& 1 \\
\end{tabular}
\caption{Pivoting su $y_{21}$. $A_1$ entra in base e $A_4$ esce.}
\label{tab:tab22}
\end{table}
Poich� $y_{21}=1$ non c'� nulla da fare su $R_2$. Applichiamo le operazioni elementari di riga al nostro tableau come segue:
\begin{align*}
R_0&\leftarrow R_0 + 4R_2; \\
R_1&\leftarrow R_1 - R_2; \\
R_3&\leftarrow R_3 - R_2
\end{align*}
Il nostro nuovo tableau diventa quindi quello in tabella \vref{tab:tab23}.
\begin{table}[htbp]
\centering
\begin{tabular}{rrcccccc}
 	  & 			&$-\varphi$ & $x_1$ & $x_2$ & $x_3$ & $x_4$ & $x_5$ \\
$R_0$ & $\OL{c_j}$ 	& \Sc{12} 	& 0 	& -1 	& 0 	& 4 	& 0 \\
\cline{3-8}
$R_1$ & $x_3$ 		& \Sc{1} 	& 0 	& 1 	& 1 	& -1 	& 0 \\
$R_2$ & $x_1$ 		& \Sc{3} 	& 1		& 1		& 0 	& 1 	& 0 \\
$R_3$ & $x_5$ 		& \Sc{0} 	& 0 	& -1 	& 0 	& -1 	& 1 \\
\end{tabular}
\caption{Secondo tableau. Vertice $\delta(3,0)$}
\label{tab:tab23}
\end{table}

Ora che $A_1$ � entrato in base e $A_4$ ne � uscito, abbiamo una nuova base $\mathcal{B}$ e una nuova BFS $x$:
\begin{align*}
\mathcal{B}&=\{A_3,A_1,A_5\} \\
x&=(3,0,1,0,0)
\end{align*}
Ci troviamo nel \textbf{vertice} $\pmb{\delta}$, ma questa non � ancora la BFS ottima. Inoltre, ci troviamo nel caso di una \textbf{base degenere}: la variabile $x_5$, che � in base, ha valore nullo. Questo non dovrebbe comunque crearci problemi in quanto stiamo applicando la regola di Bland.
L'unica colonna ad avere un $\OL{c_j}$ negativo � $A_2$ ed � su questa che cercheremo l'elemento di pivot $y_{\ell 2}$:
\begin{align*}
\vartheta_{\max}&=\min_{i:y_{i2}>0}\frac{y_{i0}}{y_{i2}}=\frac{y_{i0}}{y_{\ell 2}} \\
\vartheta_{\max}&=\min\left(\frac{1}{1},\frac{3}{1}\right)=\frac{1}{1}=\frac{y_{10}}{\pmb{y_{12}}}
\end{align*}
Faremo pivoting sull'elemento $y_{12}$ (cerchiato in tabella \vref{tab:tab24}). 
\begin{table}[htbp]
\centering
\begin{tabular}{rrcccccc}
 	  & 			&$-\varphi$ & $x_1$ & $x_2$ & $x_3$ & $x_4$ & $x_5$ \\
$R_0$ & $\OL{c_j}$ 	& \Sc{12} 	& 0 	& -1 	& 0 	& 4 	& 0 \\
\cline{3-8}
$R_1$ & $x_3$ 		& \Sc{1} 	& 0 	& \C{1}	& 1 	& -1 	& 0 \\
$R_2$ & $x_1$ 		& \Sc{3} 	& 1		& 1		& 0 	& 1 	& 0 \\
$R_3$ & $x_5$ 		& \Sc{0} 	& 0 	& -1 	& 0 	& -1 	& 1 \\
\end{tabular}
\caption{Pivoting su $y_{12}$. $A_2$ entra in base e $A_3$ esce.}
\label{tab:tab24}
\end{table}
Le operazione di pivoting saranno:
\begin{align*}
R_0&\rightarrow R_0 + R_1 \\
R_2&\rightarrow R_2 - R_1 \\
R_3&\rightarrow R_3 + R_1
\end{align*}
Il nostro nuovo tableau, quindi, � quello in tabella \vref{tab:tab25}. 
\begin{table}[htbp]
\centering
\begin{tabular}{rrcccccc}
 	  & 			&$-\varphi$ & $x_1$ & $x_2$ & $x_3$ & $x_4$ & $x_5$ \\
$R_0$ & $\OL{c_j}$ 	& \Sc{13} 	& 0 	& 0 	& 1 	& 3 	& 0 \\
\cline{3-8}
$R_1$ & $x_2$ 		& \Sc{1} 	& 0 	& 1 	& 1 	& -1 	& 0 \\
$R_2$ & $x_1$ 		& \Sc{2} 	& 1		& 0		& -1 	& 2 	& 0 \\
$R_3$ & $x_5$ 		& \Sc{1} 	& 0 	& 0 	& 1 	& -2 	& 1 \\
\end{tabular}
\caption{Terzo tableau. Vertice $\gamma(2,1)$}
\label{tab:tab25}
\end{table}
Notiamo che tutti i $\OL{c_j}$ sono non negativi, perci� ci troviamo nella \textbf{BFS ottima}. La base $\mathcal{B}$ e la soluzione $x$ sono quindi:
\begin{align*}
\mathcal{B}&={A_2,A_1,A_5} \\
x&=(2,1,0,0,1)
\end{align*}
La soluzione ottima � quella del vertice $\pmb{\gamma(2,1)}$. Riassumendo, tutti i valori delle variabili in gioco sono i seguenti:
\begin{align*}
z&=-3\varphi=-3(-13)=39 \\
x_1&=2 \\
x_2&=1
\end{align*}
\subsection{Conclusione}
La soluzione ottima consiste nel produrre $2$ lotti di composto 1 e $1$ lotto di composto 2, ottenendo un \textbf{profitto} pari a $39\,000$\euro .

\section{Esercizio 3}

Sia dato - in linguaggio naturale - il seguente problema di ottimizzazione:
\begin{enumerate}
\item Un'azienda produce due tipi di composto A e B;
\item Il profitto del composto A � il doppio di quello del composto B;
\item Per motivi di mercato non si possono produrre pi� di $2$T di composto A;
\item Ogni tonnellata di ogni composto contiene $1$Q di sostanza base;
\item Ho a disposizione $3$Q di sostanza base;
\item $1$T di composto A contiene $1$Q di sostanza chimica;
\item $1$T di composto B contiene $2$Q di sostanza chimica;
\item Ho a disposizione $5$Q di sostanza chimica.
\end{enumerate}
Si modelli il problema come un problema di programmazione lineare, lo si porti in forma standard, si realizzi una rappresentazione grafica del problema e si ottimizzi la funzione di profitto attraverso il \textbf{metodo del simplesso} affinch� \textbf{si ottenga il massimo profitto dalla produzione dei prodotti} nel rispetto dei vincoli assegnati. Si utilizzi la \textit{regola di Dantzig} per scegliere le basi su cui fare pivot.

\subsection{Modellizzazione}
Si indichi con:
\begin{itemize}
\item $x_1$ il numero di tonnellate di composto 1;
\item $x_2$ il numero di tonnellate di composto 2.
\end{itemize}
Lo scopo del nostro problema � di massimizzare i profitti ottenuti dalla produzione. Anche se non siamo a conoscenza degli esatti profitti dati da ogni prodotto, abbiamo comunque a disposizione la relazione data dalla proposizione 2, cio� la variabile $x_1$ rende il doppio della variabile $x_2$. Possiamo quindi esprimere cos� la funzione di profitto:
$$
\max z = 2x_1 + x_2
$$

Modelliamo ora i vincoli espressi dal problema.

La relazione 3 � cos� facilmente esprimibile:
$$
x_1 \leq 2
$$
Dalle relazioni 4 e 5 possiamo dedurre il seguente vincolo:
\begin{equation*}
x_1 + x_2 \leq 3
\end{equation*}
Dalle relazioni 6, 7 e 8 possiamo dedurre il seguente vincolo:
\begin{equation*}
x_1 + 2x_2 \leq 5
\end{equation*}
Infine imponiamo il vincolo, implicito, che la produzione non pu� essere negativa:
$$
x_1,x_2 \geq0
$$
Il modello matematico pu� essere quindi cos� riassunto (sono state apportate semplificazione algebriche):
\begin{align*}
\max z	&= 2x_1+x_2 \\
\st\;\; & x_1 \leq 2\\
	  	& x_1+x_2 \leq 3\\
		& x_1+2x_2 \leq 5\\
	  	& x_1,x_2 \geq 0
\end{align*}

\subsection{Problema in forma grafica}

In figura \vref{fig:graph3} � rappresentato graficamente il problema presentato. In giallo � rappresentato il politopo $P$ e sono stati chiamati $\alpha,\beta,\gamma,\delta,\varepsilon$ i suoi cinque vertici, i quali sappiamo corrispondere ognuno ad una BFS.
Il gradiente della funzione obiettivo vale
\begin{equation*}
\nabla(z)=\left(\frac{\partial z}{\partial x},\frac{\partial z}{\partial y}\right) = \left(2,1\right) \\
\end{equation*}
Il politopo $P$ �, ovviamente, limitato nella direzione del gradiente (si fa notare che finora $P$ � sempre limitato in ogni direzione, quindi qualsiasi direzione avesse il gradiente non ci sarebbero problemi).

\begin{figure}[htbp]
\centering
\begin{tikzpicture}
\begin{axis}
[axis lines=middle, axis equal, enlargelimits, xlabel=$x_1$, ylabel=$x_2$,
 every axis x label/.style={
    at={(ticklabel* cs:1.01)},
    anchor=west,
 },
 every axis y label/.style={
    at={(ticklabel* cs:1.01)},
    anchor=south,
 },]
    \path[name path=AX] 
        (axis cs:\pgfkeysvalueof{/pgfplots/xmin},0)--
        (axis cs:\pgfkeysvalueof{/pgfplots/xmax},0);
    \path[name path=AY] 
        (axis cs:0,\pgfkeysvalueof{/pgfplots/ymin})--
        (axis cs:0,\pgfkeysvalueof{/pgfplots/ymax});
    \path[name path=UP]
    	(axis cs:\pgfkeysvalueof{/pgfplots/xmin},\pgfkeysvalueof{/pgfplots/ymax})--
    	(axis cs:\pgfkeysvalueof{/pgfplots/xmax},\pgfkeysvalueof{/pgfplots/ymax});
\addplot
[domain=0:4, samples=10, thick, blue, name path=x2y5]
{-.5*x+2.5} node [pos=0.8,pin={75:{\color{blue}$x_1+2x_2=5$}}, inner sep=0pt] {};
\addplot
[domain=0:3, samples=10, thick, red, name path=xy3]
{-x+3} node [pos=0.2, pin={85:{\color{red}$x_1+x_2=3$}}, inner sep=0pt] {};
\addplot
[domain=0:5, samples = 10, thick, purple, name path=x2]
(2,x) node [pos=0.5, anchor=north, pin={0:{\color{purple}$x_1=2$}}, inner sep= 0pt] {};
\addplot[thick, fill=yellow, fill opacity=0.5] fill between [of=x2y5 and AX, soft clip={domain=0:3}];
\addplot[white] fill between [of=xy3 and UP];
%\addplot[pattern=north east lines, pattern color=red!10] fill between [reverse=true, of=AX and UP, soft clip={domain=0:5}];
\addplot[white] fill between [of=x2 and AX];
\addplot[pattern=north east lines, pattern color=red!10] fill between [of=xy3 and AX];
\addplot[pattern=vertical lines, pattern color=blue!10] fill between [of=x2y5 and AX];
%\addplot[pattern=north east lines, pattern color=blue!10] fill between [of=AX and 2x2y9, soft clip={domain=0:5}];
\addplot[pattern=horizontal lines, pattern color=purple!10] fill between [of=AY and x2];
\intse{AX}{AY}{$\alpha$}{alp};
\intse{AY}{x2y5}{$\beta$}{bet};
\intne{x2y5}{xy3}{$\gamma$}{gam};
\inte{xy3}{x2}{$\delta$}{del};
\intne{x2}{AX}{$\varepsilon$}{eps};
\node at (axis cs:1,1) {$P$};
\addplot[-latex, thick] coordinates
           {(0,0) (2/2.24,1/2.24)} node [pos=.7, anchor=south, label={0:{\small $\nabla z$}}] {};
\end{axis}
\end{tikzpicture}
\caption{Rappresentazione cartesiana del problema di programmazione lineare}
\label{fig:graph3}
\end{figure}

\subsection{Forma standard}
Ricordiamo che un problema di \textbf{programmazione lineare in forma standard} � nella forma (matriciale):
\begin{align*}
\min c'x& \\
Ax& = b \\
x& \geq 0
\end{align*}
Trasformiamo la funzione obiettivo $z$ in $\varphi$ tale che:
\begin{equation*}
\varphi=-z=-2x_1-x_2
\end{equation*}
Quindi introduciamo una \textbf{variabile slack} per ogni disequazione con simbolo $\leq$. Otterremo infine:
\begin{alignat*}{7}
&\min \varphi = \quad && -2x_1 \quad\; && -x_2 \quad\;\; && \qquad\qquad && \qquad\qquad && \qquad\qquad && \\
&\;\st  &&+x_1			&&		 		&&+\pmb{x_3}	&&		 		&&					&&=2\\
&	 	&&+x_1			&&+x_2			&&				&& +\pmb{x_4}	&&					&&=3\\
&	 	&&+x_1			&&+2x_2			&&				&&				&&+\pmb{x_5}		&&=5\\
&		&&\quad\; x_1,	&&\quad\; x_2,	&&\quad\; x_3,	&&\quad\; x_4,	&&\quad\; x_5		&&\geq 0
\end{alignat*}

\subsection{Risoluzione tramite tableau}

\begin{table}[htbp]
\centering
\begin{tabular}{rcccccc}
			&$-\varphi$ & $x_1$ & $x_2$ & $x_3$ & $x_4$ & $x_5$ \\
$\OL{c_j}$ 	& \Sc{0} 	& -2 	& -1 	& 0 	& 0 	& 0 \\
\cline{2-7}
$x_3$ 		& \Sc{2} 	& 1 	& 0 	& 1 	& 0 	& 0 \\
$x_4$ 		& \Sc{3} 	& 1 	& 1		& 0 	& 1 	& 0 \\
$x_5$ 		& \Sc{5} 	& 1 	& 2 	& 0 	& 0 	& 1 \\
\end{tabular}
\caption{Tableau iniziale. Vertice $\alpha(0,0)$}
\label{tab:tab31}
\end{table}

In tabella \vref{tab:tab31} il tableau ricavato dal nostro problema. Si noti che le ultime 3 colonne formano gi� una matrice identit�, perci� le assumeremo come base.
\begin{align*}
\mathcal{B}&=\{A_3,A_4,A_5\}\\
x&=(0,0,2,3,5)
\end{align*}
Ci troviamo nell'origine, che appartiene al politopo $P$ trovato in precedenza e che in particolare � il \textbf{vertice} $\pmb{\alpha}$.
Poich� non tutti i $\OL{c_j}$ sono non negativi, la nostra non � la BFS ottima e dobbiamo muoverci in una BFS migliore. Applicando la \textbf{regola di Dantzig}, facciamo entrare in base la colonna il cui $\OL{c_j}$ � maggiore in valore assoluto. Nel nostro caso, prenderemo in considerazione quindi la colonna $\pmb{A_1}$.
Per scegliere su quale elemento fare \textbf{pivoting}, dobbiamo ottenere il valore di $y_{\ell 1}$ tale che:
$$
\vartheta_{\max}=\min_{i:y_{i1}>0}\frac{y_{i0}}{y_{i1}}=\frac{y_{i0}}{y_{\ell 1}}
$$
Perci�, operando con gli elementi nel tableau:
\begin{align*}
\vartheta_{\max}=\min\left(\frac{2}{1},\frac{3}{1},\frac{5}{1}\right)=\frac{2}{1}=\frac{y_{10}}{\pmb{y_{11}}}
\end{align*}
Faremo pivoting sull'elemento $y_{11}$ (cerchiato in tabella \vref{tab:tab32}). Il nostro scopo � ora far comparire uno 0 nella colonna dell'elemento pivot in tutte le righe tranne quella in cui si trova l'elemento pivot e far comparire un 1 in quest'ultima.
\begin{table}[htbp]
\centering
\begin{tabular}{rrcccccc}
 	  & 			&$-\varphi$ & $x_1$ & $x_2$ & $x_3$ & $x_4$ & $x_5$ \\
$R_0$ & $\OL{c_j}$ 	& \Sc{0} 	& -2 	& -1 	& 0 	& 0 	& 0 \\
\cline{3-8}
$R_1$ & $x_3$ 		& \Sc{2} 	& \C{1}	& 0 	& 1 	& 0 	& 0 \\
$R_2$ & $x_4$ 		& \Sc{3} 	& 1		& 1		& 0 	& 1 	& 0 \\
$R_3$ & $x_5$ 		& \Sc{5} 	& 1 	& 2 	& 0 	& 0 	& 1 \\
\end{tabular}
\caption{Pivoting su $y_{11}$. $A_1$ entra in base e $A_3$ esce.}
\label{tab:tab32}
\end{table}
Poich� $y_{11}=1$ non c'� nulla da fare su $R_1$. Applichiamo le operazioni elementari di riga al nostro tableau come segue:
\begin{align*}
R_0&\leftarrow R_0 + 2R_2; \\
R_2&\leftarrow R_2 - R_1; \\
R_3&\leftarrow R_3 - R_1.
\end{align*}
Il nostro nuovo tableau diventa quindi quello in tabella \vref{tab:tab33}.
\begin{table}[htbp]
\centering
\begin{tabular}{rrcccccc}
 	  & 			&$-\varphi$ & $x_1$ & $x_2$ & $x_3$ & $x_4$ & $x_5$ \\
$R_0$ & $\OL{c_j}$ 	& \Sc{4} 	& 0 	& -1 	& 2 	& 0 	& 0 \\
\cline{3-8}
$R_1$ & $x_1$ 		& \Sc{2} 	& 1 	& 0 	& 1 	& 0 	& 0 \\
$R_2$ & $x_4$ 		& \Sc{1} 	& 0		& 1		& -1 	& 1 	& 0 \\
$R_3$ & $x_5$ 		& \Sc{3} 	& 0 	& 2 	& -1 	& 0 	& 1 \\
\end{tabular}
\caption{Secondo tableau. Vertice $\varepsilon(2,0)$}
\label{tab:tab33}
\end{table}

Ora che $A_1$ � entrato in base e $A_3$ ne � uscito, abbiamo una nuova base $\mathcal{B}$ e una nuova BFS $x$:
\begin{align*}
\mathcal{B}&=\{A_1,A_4,A_5\} \\
x&=(2,0,0,1,3)
\end{align*}
Ci troviamo nel \textbf{vertice} $\pmb{\varepsilon}$, ma questa non � ancora la BFS ottima.
L'unica colonna ad avere un $\OL{c_j}$ negativo � $A_2$ ed � su questa che cercheremo l'elemento di pivot $y_{\ell 2}$:
\begin{align*}
\vartheta_{\max}&=\min_{i:y_{i2}>0}\frac{y_{i0}}{y_{i2}}=\frac{y_{i0}}{y_{\ell 2}} \\
\vartheta_{\max}&=\min\left(\frac{1}{1},\frac{3}{2}\right)=\frac{1}{1}=\frac{y_{20}}{\pmb{y_{22}}}
\end{align*}
Faremo pivoting sull'elemento $y_{22}$ (cerchiato in tabella \vref{tab:tab34}). 
\begin{table}[htbp]
\centering
\begin{tabular}{rrcccccc}
 	  & 			&$-\varphi$ & $x_1$ & $x_2$ & $x_3$ & $x_4$ & $x_5$ \\
$R_0$ & $\OL{c_j}$ 	& \Sc{4} 	& 0 	& -1 	& 2 	& 0 	& 0 \\
\cline{3-8}
$R_1$ & $x_1$ 		& \Sc{2} 	& 1 	& 0 	& 1 	& 0 	& 0 \\
$R_2$ & $x_4$ 		& \Sc{1} 	& 0		& \C{1}	& -1 	& 1 	& 0 \\
$R_3$ & $x_5$ 		& \Sc{3} 	& 0 	& 2 	& -1 	& 0 	& 1 \\
\end{tabular}
\caption{Pivoting su $y_{22}$. $A_2$ entra in base e $A_4$ esce.}
\label{tab:tab34}
\end{table}
Le operazioni di pivoting saranno:
\begin{align*}
R_0&\rightarrow R_0 + R_2 \\
R_3&\rightarrow R_3 - 2R_2
\end{align*}
Il nostro nuovo tableau, quindi, � quello in tabella \vref{tab:tab35}. 
\begin{table}[htbp]
\centering
\begin{tabular}{rrcccccc}
 	  & 			&$-\varphi$ & $x_1$ & $x_2$ & $x_3$ & $x_4$ & $x_5$ \\
$R_0$ & $\OL{c_j}$ 	& \Sc{5} 	& 0 	& 0 	& 1 	& 1 	& 0 \\
\cline{3-8}
$R_1$ & $x_1$ 		& \Sc{2} 	& 1 	& 0 	& 1 	& 0 	& 0 \\
$R_2$ & $x_2$ 		& \Sc{1} 	& 0		& \C{1}	& -1 	& 1 	& 0 \\
$R_3$ & $x_5$ 		& \Sc{1} 	& 0 	& 0 	& 1 	& -2 	& 1 \\
\end{tabular}
\caption{Terzo tableau. Vertice $\delta(2,1)$}
\label{tab:tab35}
\end{table}
Notiamo che tutti i $\OL{c_j}$ sono non negativi, perci� ci troviamo nella \textbf{BFS ottima}. La base $\mathcal{B}$ e la soluzione $x$ sono quindi:
\begin{align*}
\mathcal{B}&={A_2,A_1,A_5} \\
x&=(2,1,0,0,1)
\end{align*}
La soluzione ottima � quella del vertice $\pmb{\delta(2,1)}$. Riassumendo, tutti i valori delle variabili in gioco sono i seguenti:
\begin{align*}
z&=-\varphi=5 \\
x_1&=2 \\
x_2&=1
\end{align*}
\subsection{Conclusione}
La soluzione ottima consiste nel produrre $2$T di composto A e $1$T di composto B ottenendo un \textbf{profitto} pari a 5 volte il profitto di $1$T di composto B.


\chapter{31/03/2014}

I problemi saranno posti in maniera leggermente diversa, cio� quella fornita sul pdf reperibile sul sito del docente al seguente link: \url{http://www.or.deis.unibo.it/staff_pages/martello/testi_esercizi_ottimizzazione.pdf}.
Inoltre, anche se durante l'esercitazione non � stata trovata la soluzione dei problemi duali, dato che il metodo per individuarli � stato spiegato dal prof. nella lezione subito successiva, ho ritenuto opportuno e interessante cercarle io stesso e inserirle in questo eserciziario. A maggior ragione, le soluzioni dei duali \textbf{potrebbero essere errate}, per cui chiedo ad ognuno di provare a rivederle e comunicarmi gli eventuali errori trovati.
Inoltre, ho deciso - in maniera del tutto personale e arbitraria - di preporre la rappresentazione grafica alla risoluzione con tableau negli esercizi di ottimizzazione. L'unico motivo � che mi piace avere un'idea un po' pi� concreta di quello che sta succedendo sul piano geometrico.

\section{Esercizio 1}
Un'azienda chimica produce due tipi di composto, A e B, che danno lo stesso profitto, utilizzando una sostanza base della quale sono disponibili 8 quintali. Ogni tonnellata di composto (indipendentemente dal tipo) contiene un quintale di sostanza base. Il numero di tonnellate di composto A prodotto deve superare di almeno una unit� il numero di tonnellate di composto B prodotto. Per problemi di stoccaggio non si possono produrre pi� di 6 tonnellate di composto A. Si associ la variabile $x_1$ al composto A e la variabile $x_2$ al composto B.
\begin{enumerate}
\item Definire il modello LP che determina la funzione di massimo profitto.
\item Porre il modello in forma standard e risolverlo con il metodo delle due fasi e la regola di Bland, introducendo il minimo numero di variabili artificiali. Dire esplicitamente qual � la soluzione trovata.
\item Disegnare con cura la regione ammissibile.
\item Costruire il duale del modello definito al punto 2 e ricavarne le soluzioni ottime.
\item Imporre il vincolo di interezza sulle variabili (supporre che non si possano produrre frazioni di tonnellate) e risolvere il problema con il metodo branch-and-bound. [\textit{Questo punto non sar� analizzato perch� in data di stesura del documento (04/04/2014) l'argomento non � ancora stato trattato dal prof}]
\end{enumerate}

\subsection{Modellizzazione}

Si indichi con:
\begin{itemize}
\item $x_1$ il numero di tonnellate di composto A;
\item $x_2$ il numero di tonnellate di composto B.
\end{itemize}
Lo scopo del nostro problema � di massimizzare i profitti ottenuti dalla produzione. Anche se non siamo a conoscenza degli esatti profitti dati da ogni prodotto, sappiamo che entrambi i composti portano allo stesso profitto. Possiamo quindi esprimere cos� la funzione di profitto:
$$
\max z = x_1 + x_2
$$

Modelliamo ora i vincoli espressi dal problema.
Il modello matematico pu� essere quindi cos� riassunto (sono state apportate semplificazione algebriche):
\begin{align*}
\max z	&= 2x_1+x_2 \\
\st\;\;	& x_1+x_2 \leq 8\\
		& x_1 \geq x_2 + 1\\
		& x_1 \leq 6 \\
	  	& x_1,x_2 \geq 0
\end{align*}

\subsection{Problema in forma grafica}

In figura \vref{fig:graph4} � rappresentato graficamente il problema presentato. In giallo � rappresentato il politopo $P$ e sono stati chiamati $\alpha,\beta,\gamma,\delta$ i suoi cinque quattro, i quali sappiamo corrispondere ognuno ad una BFS.
Il gradiente della funzione obiettivo vale
\begin{equation*}
\nabla(z)=\left(\frac{\partial z}{\partial x},\frac{\partial z}{\partial y}\right) = \left(1,1\right) \\
\end{equation*}
Il politopo $P$ �, ovviamente, limitato nella direzione del gradiente (si fa notare che finora $P$ � sempre limitato in ogni direzione, quindi qualsiasi direzione avesse il gradiente non ci sarebbero problemi).

\begin{figure}[htbp]
\centering
\begin{tikzpicture}
\begin{axis}
[axis lines=middle, axis equal, enlargelimits, xlabel=$x_1$, ylabel=$x_2$,
 every axis x label/.style={
    at={(ticklabel* cs:1.01)},
    anchor=west,
 },
 every axis y label/.style={
    at={(ticklabel* cs:1.01)},
    anchor=south,
 },]
    \path[name path=AX] 
        (axis cs:\pgfkeysvalueof{/pgfplots/xmin},0)--
        (axis cs:\pgfkeysvalueof{/pgfplots/xmax},0);
    \path[name path=AY] 
        (axis cs:0,\pgfkeysvalueof{/pgfplots/ymin})--
        (axis cs:0,\pgfkeysvalueof{/pgfplots/ymax});
    \path[name path=UP]
    	(axis cs:\pgfkeysvalueof{/pgfplots/xmin},\pgfkeysvalueof{/pgfplots/ymax})--
    	(axis cs:\pgfkeysvalueof{/pgfplots/xmax},\pgfkeysvalueof{/pgfplots/ymax});
\addplot
[domain=0:8, samples=10, thick, blue, name path=xy8]
{-x+8} node [pos=0.2,pin={75:{\color{blue}$x_1+x_2=8$}}, inner sep=0pt] {};
\addplot
[domain=0:8, samples=10, thick, red, name path=xy1]
{x-1} node [pos=0.8, pin={-85:{\color{red}$x_1=x_2+1$}}, inner sep=0pt] {};
\addplot
[domain=0:8, samples = 10, thick, purple, name path=x6]
(6,x) node [pos=0.3, anchor=north, pin={0:{\color{purple}$x_1=6$}}, inner sep= 0pt] {};
\addplot[thick, fill=yellow, fill opacity=0.5] fill between [of=xy1 and AX, soft clip={domain=1:6}];
\addplot[white] fill between [of=xy8 and UP];
%\addplot[pattern=north east lines, pattern color=red!10] fill between [reverse=true, of=AX and UP, soft clip={domain=0:5}];
\addplot[white] fill between [of=x6 and AX];
\addplot[pattern=north east lines, pattern color=blue!10] fill between [of=xy8 and AX];
\addplot[pattern=north west lines, pattern color=red!10] fill between [of=xy1 and AX];
%\addplot[pattern=north east lines, pattern color=blue!10] fill between [of=AX and 2x2y9, soft clip={domain=0:5}];
\addplot[pattern=horizontal lines, pattern color=purple!10] fill between [of=AY and x6];
\ints{AX}{xy1}{$\alpha$}{alp};
\ints{xy1}{xy8}{$\beta$}{bet};
\intw{xy8}{x6}{$\gamma$}{gam};
\intnw{x6}{AX}{$\delta$}{del};
\node at (axis cs:4.5,1.5) {$P$};
\addplot[-latex, thick] coordinates
           {(0,0) (1/1.414,1/1.414)} node [pos=.3, anchor=south, label={45:{\small $\nabla z$}}] {};
\end{axis}
\end{tikzpicture}
\caption{Rappresentazione cartesiana del problema di programmazione lineare}
\label{fig:graph4}
\end{figure}

Possiamo osservare che anche solo dal grafico � facilmente intuibile dove si trover� la soluzione ottima. Il gradiente $\nabla z$ � \textbf{perpendicolare} allo spigolo $\OL{\beta \gamma}$, da ci� potremmo dedurre che non esiste una soluzione ottima, ma che ve ne sono infinite e tutte posizionate su questo spigolo. Riprenderemo questa considerazione in seguito, dopo aver risolto il problema con il metodo del simplesso.

\subsection{Forma standard}
Ricordiamo che un problema di \textbf{programmazione lineare in forma standard} � nella forma (matriciale):
\begin{align*}
\min c'x& \\
Ax& = b \\
x& \geq 0
\end{align*}
Trasformiamo la funzione obiettivo $z$ in $\varphi$ tale che:
\begin{equation*}
\varphi=-z=-x_1-x_2
\end{equation*}
Quindi introduciamo una \textbf{variabile slack} per ogni disequazione con simbolo $\leq$ e una \textbf{variabile surplus} per ogni disequazione con simbolo $\geq$. Otterremo infine:
\begin{alignat*}{7}
&\min \varphi = \quad && -x_1 \quad\; && -x_2 \quad\;\; && \qquad\qquad && \qquad\qquad && \qquad\qquad && \\
&\;\st  &&+x_1			&&+x_2	 		&&+\pmb{x_3}	&&		 		&&					&&=8\\
&	 	&&+x_1			&&-x_2			&&				&& -\pmb{x_4}	&&					&&=1\\
&	 	&&+x_1			&&				&&				&&				&&+\pmb{x_5}		&&=6\\
&		&&\quad\; x_1,	&&\quad\; x_2,	&&\quad\; x_3,	&&\quad\; x_4,	&&\quad\; x_5		&&\geq 0
\end{alignat*}

\subsection{Risoluzione tramite tableau}

\begin{table}[htbp]
\centering
\begin{tabular}{rcccccc}
			&$-\varphi$ & $x_1$ & $x_2$ & $x_3$ & $x_4$ & $x_5$ \\
$\OL{c_j}$ 	& \Sc{0} 	& -1 	& -1 	& 0 	& 0 	& 0 \\
\cline{2-7}
$R_1$ 		& \Sc{8} 	& 1 	& 1 	& 1 	& 0 	& 0 \\
$R_2$		& \Sc{1} 	& 1 	& -1	& 0 	& -1 	& 0 \\
$R_3$		& \Sc{6} 	& 1 	& 0 	& 0 	& 0 	& 1 \\
\end{tabular}
\caption{Tableau iniziale.}
\label{tab:tab41}
\end{table}

In tabella \vref{tab:tab41} il tableau ricavato dal nostro problema. A differenza dei precedenti esercizi, la fortuna non � dalla nostra parte e non abbiamo nessuna sottomatrice identit� a disposizione da utilizzare come base ammissibile.
Si potrebbe \textit{erroneamente} pensare che per ottenere una BFS sia sufficiente operare $R_2\leftarrow -1\cdot R_2$. Ma si fa subito notare che cos� facendo otterremo come base:
\begin{align*}
\mathcal{B}&=\{A_3,A_4,A_5\}\\
x&=(0,0,8,1,6)
\end{align*}
Questa \textbf{non � una BFS} in quanto ricade \textit{all'esterno} del politopo $P$. Per ottenere una BFS di partenza, quindi, ricorriamo alla \textbf{fase 1 del metodo del simplesso}.

\subsubsection{Fase 1 - aggiunta di variabili artificiali}

Per ottenere una BFS aggiungiamo un numero $n'\leq m$ di variabili artificiali tali da riuscire ad ottenere una BFS nel nuovo problema con $m$ vincoli e $n+n'$ variabili. Ipoteticamente, potremmo aggiungere sempre $n'=m$ variabili artificiali tali da formare gi� loro una sottomatrice identit� nel tableau, ma tale metodo risulterebbe molto sconveniente nel caso in cui i vincoli e le variabili fossero centinaia o migliaia. Inoltre, ma non meno importante, la traccia dell'esercizio richiede esplicitamente di \textbf{introdurre il minore numero di variabili artificiali}.

Per ridurre al minimo le variabili artificiali $x_i^a,\quad i=1,\cdots,n'$ � sufficiente aggiungerne una per ogni colonna della matrice identit� mancante nel tableau originale. Nel nostro caso manca solo la seconda colonna e sar� quella che introdurremo con l'\textit{unica} variabile artificiale $x^a$, trasformando il secondo vincolo in:
$$
x_1 - x_2 - x_4 + x^a = 1
$$
Il nostro scopo, dopo l'introduzione di $x^a$, sar� quello di \textbf{eliminarla} dalla base. Per far ci� bisogna fare in modo che questa valga zero e quindi introduciamo, a tale scopo, una nuova funzione obiettivo da minimizzare $\psi$ tale che:
$$
\psi = \sum_{i=1}^{n'}x_i^a = x^a
$$
Scriviamo il nuovo tableau in tabella \vref{tab:tab42} e applichiamo il simplesso per ottimizzare la nostra funzione $\psi$.
\begin{table}[htbp]
\centering
\begin{tabular}{rrccccccc}
 	  & 			&$-\psi$	& $x_1$ & $x_2$ & $x_3$ & $x_4$ & $x_5$	& $x^a$\\
$R_0$ & $\OL{c_j}$ 	& \Sc{0} 	& 0 	& 0 	& 0 	& 0 	& \Sc{0}& 1\\
\cline{3-9}
$R_1$ & $x_3$ 		& \Sc{8} 	& 1		& 1 	& 1 	& 0 	& \Sc{0}& 0 \\
$R_2$ & $x^a$ 		& \Sc{1} 	& 1		& -1	& 0 	& -1 	& \Sc{0}& 1 \\
$R_3$ & $x_5$ 		& \Sc{6} 	& 1 	& 0 	& 0 	& 0 	& \Sc{1}& 0 \\
\end{tabular}
\caption{Nuovo tableau con la variabile artificiale $x^a$.}
\label{tab:tab42}
\end{table}
Abbiamo una sottomatrice identit� formata dalla base:
$$
\mathcal{B}={A_3,A_6,A_5}
$$
Per avere a avere a disposizione i valori delle coordinate della BFS del nuovo problema, � necessario che:
$$
y_{ij}=0 \quad \forall i,j:A_j\in\mathcal{B},i\neq j
$$
Condizione vera per ogni valore tranne $y_{06}$ che provvediamo ad annullare tramite l'operazione elementare di riga:
$$
R_0\leftarrow R_0 - R_2
$$
Nel nuovo tableau in figura \vref{tab:tab43} faremo pivoting sull'unica colonna con $\OL{c_j}<0$, cio� su $A_1$.
Per scegliere su quale elemento fare \textbf{pivoting}, dobbiamo ottenere il valore di $y_{\ell 1}$ tale che:
$$
\vartheta_{\max}=\min_{i:y_{i1}>0}\frac{y_{i0}}{y_{i1}}=\frac{y_{i0}}{y_{\ell 1}}
$$
Perci�, operando con gli elementi nel tableau:
$$
\vartheta_{\max}=\min\left(\frac{8}{1},\frac{1}{1},\frac{6}{1}\right)=\frac{1}{1}=\frac{y_{20}}{\pmb{y_{21}}}
$$
Faremo pivoting sull'elemento $y_{21}$ (cerchiato in tabella). Il nostro scopo � ora far comparire uno 0 nella colonna dell'elemento pivot in tutte le righe tranne quella in cui si trova l'elemento pivot e far comparire un 1 in quest'ultima.
\begin{table}[htbp]
\centering
\begin{tabular}{rrccccccc}
 	  & 			&$-\psi$	& $x_1$ & $x_2$ & $x_3$ & $x_4$ & $x_5$	& $x^a$\\
$R_0$ & $\OL{c_j}$ 	& \Sc{-1} 	& -1 	& 1 	& 0 	& 1 	& \Sc{0}& 0\\
\cline{3-9}
$R_1$ & $x_3$ 		& \Sc{8} 	& 1		& 1 	& 1 	& 0 	& \Sc{0}& 0 \\
$R_2$ & $x^a$ 		& \Sc{1} 	& \C{1} & -1	& 0 	& -1 	& \Sc{0}& 1 \\
$R_3$ & $x_5$ 		& \Sc{6} 	& 1 	& 0 	& 0 	& 0 	& \Sc{1}& 0 \\
\end{tabular}
\caption{Pivoting su $y_21$. $A_1$ entra in base e $A_6$ esce.}
\label{tab:tab43}
\end{table}
Poich� $y_{21}=1$ non c'� nulla da fare su $R_2$. Applichiamo le operazioni elementari di riga al nostro tableau come segue:
\begin{align*}
R_0&\leftarrow R_0 + R_2; \\
R_1&\leftarrow R_1 - R_2; \\
R_3&\leftarrow R_3 - R_2.
\end{align*}
Il nostro nuovo tableau diventa quindi quello in tabella \vref{tab:tab44}.
\begin{table}[htbp]
\centering
\begin{tabular}{rrccccccc}
 	  & 			&$-\psi$	& $x_1$ & $x_2$ & $x_3$ & $x_4$ & $x_5$	& $x^a$\\
$R_0$ & $\OL{c_j}$ 	& \Sc{0} 	& 0 	& 0 	& 0 	& 0 	& \Sc{0}& 1\\
\cline{3-9}
$R_1$ & $x_3$ 		& \Sc{7} 	& 0		& 2 	& 1 	& 1 	& \Sc{0}& -1 \\
$R_2$ & $x_1$ 		& \Sc{1} 	& 1		& -1	& 0 	& -1 	& \Sc{0}& 1 \\
$R_3$ & $x_5$ 		& \Sc{5} 	& 0 	& 1 	& 0 	& 1 	& \Sc{1}& -1 \\
\end{tabular}
\caption{Secondo tableau. Vertice $\alpha(1,0)$}
\label{tab:tab44}
\end{table}
Siamo giunti alla soluzione ottima, essendo $\OL{c_j}>0 \quad\forall j$. Inoltre la variabile artificiale $x^a$ non � pi� in base. La nuova base e la nuova soluzione sono:
\begin{align*}
\mathcal{B}&=\{A_3,A_1,A_5\} \\
x&=(1,0,7,0,5,0)
\end{align*}
Siamo nel vertice $\alpha(1,0)$ e quindi in una BFS da cui possiamo partire per la \textbf{fase 2} del metodo del simplesso.
\subsubsection{Fase 2 - Simplesso}
Per questa fase useremo come tableau di partenza quello in tabella \vref{tab:tab44} sostituendo la funzione obiettivo fittizia $\psi$ utilizzata in precedenza con la nostra vera funzione obiettivo $\varphi$. Manterremo la variabile artificiale (che si fa notare non cambia in alcun modo il nostro problema in quanto non faremo mai entrare in base) perch�, come vedremo poi, il suo costo relativo finale sar� utile ai fini della soluzione del problema duale.
Il tableau cos� ottenuto � quello in tabella \vref{tab:tab45}
\begin{table}[htbp]
\centering
\begin{tabular}{rrccccccc}
 	  & 			&$-\varphi$	& $x_1$ & $x_2$ & $x_3$ & $x_4$ & $x_5$	& $x^a$\\
$R_0$ & $\OL{c_j}$ 	& \Sc{0} 	& -1 	& -1 	& 0 	& 0 	& \Sc{0}& 0\\
\cline{3-9}
$R_1$ & $x_3$ 		& \Sc{7} 	& 0		& 2 	& 1 	& 1 	& \Sc{0}& -1 \\
$R_2$ & $x_1$ 		& \Sc{1} 	& 1		& -1	& 0 	& -1 	& \Sc{0}& 1 \\
$R_3$ & $x_5$ 		& \Sc{5} 	& 0 	& 1 	& 0 	& 1 	& \Sc{1}& -1 \\
\end{tabular}
\caption{Secondo tableau. Vertice $\alpha(1,0)$ e funzione obiettivo $\varphi$.}
\label{tab:tab45}
\end{table}
Per applicare il simplesso, dobbiamo fare in modo che:
$$
y_{ij}=0 \quad\forall i,j:j\in\mathcal{B}, i\neq j
$$
L'elemento $y_{01}$ � l'unico a non essere nullo. Ovviamo al problema con l'operazione di riga:
$$
R_0\leftarrow R_0 + R_1
$$
Otteniamo quindi il tableau in tabella \vref{tab:tab46}.
\begin{table}[htbp]
\centering
\begin{tabular}{rrccccccc}
 	  & 			&$-\varphi$	& $x_1$ & $x_2$ & $x_3$ & $x_4$ & $x_5$	& $x^a$\\
$R_0$ & $\OL{c_j}$ 	& \Sc{1} 	& 0 	& -2 	& 0 	& -1 	& \Sc{0}& 1\\
\cline{3-9}
$R_1$ & $x_3$ 		& \Sc{7} 	& 0		& 2 	& 1 	& 1 	& \Sc{0}& -1 \\
$R_2$ & $x_1$ 		& \Sc{1} 	& 1		& -1	& 0 	& -1 	& \Sc{0}& 1 \\
$R_3$ & $x_5$ 		& \Sc{5} 	& 0 	& 1 	& 0 	& 1 	& \Sc{1}& -1 \\
\end{tabular}
\caption{Secondo tableau. Vertice $\alpha(1,0)$}
\label{tab:tab46}
\end{table}
Per fare pivoting sceglieremo la colonna $A_2$ in base alla regola di Bland (avremmo scelto la stessa colonna anche con la regola di Dantzig). Cerchiamo quindi l'elemento pivot $y_{\ell 2}$.
\begin{align*}
\vartheta_{\max}&=\min_{i:y_{i2}>0}\frac{y_{i0}}{y_{i2}}=\frac{y_{i0}}{y_{\ell 2}} \\
\vartheta_{\max}&=\min\left(\frac{7}{2},\frac{5}{1}\right)=\frac{7}{2}=\frac{y_{10}}{\pmb{y_{12}}}
\end{align*}
Faremo pivoting sull'elemento $y_{12}$ (cerchiato in tabella \vref{tab:tab47}). 
\begin{table}[htbp]
\centering
\begin{tabular}{rrccccccc}
 	  & 			&$-\varphi$	& $x_1$ & $x_2$ & $x_3$ & $x_4$ & $x_5$	& $x^a$\\
$R_0$ & $\OL{c_j}$ 	& \Sc{1} 	& 0 	& -2 	& 0 	& -1 	& \Sc{0}& 1\\
\cline{3-9}
$R_1$ & $x_3$ 		& \Sc{7} 	& 0		& \C{2}	& 1 	& 1 	& \Sc{0}& -1 \\
$R_2$ & $x_1$ 		& \Sc{1} 	& 1		& -1	& 0 	& -1 	& \Sc{0}& 1 \\
$R_3$ & $x_5$ 		& \Sc{5} 	& 0 	& 1 	& 0 	& 1 	& \Sc{1}& -1 \\
\end{tabular}
\caption{Terzo tableau. Vertice $\alpha(1,0)$}
\label{tab:tab47}
\end{table}
Le operazioni elementari di riga, \textbf{in ordine}, sono:
\begin{align*}
R_0&\leftarrow R_0 + R_1 \\
R_1&\leftarrow \frac{R_1}{2} \\
R_2&\leftarrow R_2 + R_1 \\
R_3&\leftarrow R_3 - R_1
\end{align*}
Otterremo il tableau in tabella \vref{tab:tab48}.
\begin{table}[htbp]
\centering
{
	\newcommand{\sm}{$\frac{7}{2}$}
	\newcommand{\nm}{$\frac{9}{2}$}
	\newcommand{\um}{$\frac{1}{2}$}
	\newcommand{\tm}{$\frac{3}{2}$}
\begin{tabular}{rrccccccc}
 	  & 			&$-\varphi$	& $x_1$ & $x_2$ & $x_3$ & $x_4$ & $x_5$	& $x^a$\\
$R_0$ & $\OL{c_j}$ 	& \Sc{8} 	& 0 	& 0 	& 1 	& 0 	& \Sc{0}& 0\\
\cline{3-9}
$R_1$ & $x_2$ 		& \Sc{\sm}	& 0		& 1		& \um 	& \um 	& \Sc{0}& -\um \\
$R_2$ & $x_1$ 		& \Sc{\nm} 	& 1		& 0		& \um	& -\um 	& \Sc{0}& \um \\
$R_3$ & $x_5$ 		& \Sc{-\tm}	& 0 	& 0 	& -\um 	& \um 	& \Sc{1}& -\um \\
\end{tabular}
}
\caption{Quarto tableau. Vertice $\beta(\frac{9}{2},\frac{7}{2})$. $A_2$ entra in base al posto di $A_3$, che esce.}
\label{tab:tab48}
\end{table}

Siamo giunti alla soluzione ottima, essendo $\OL{c_j}>0 \quad\forall j$. La nuova base e la nuova soluzione sono:
\begin{align*}
\mathcal{B}&=\{A_2,A_1,A_5\} \\
x&=(1,0,7,0,5,0)
\end{align*}
Siamo nel vertice $\beta(\frac{9}{2},\frac{7}{2})$ ed appartiene, come previsto durante l'analisi geometrica, allo spigolo $\OL{\beta\gamma}$. Il valore della soluzione ottima � $\varphi=-8$, proviamo ora a calcolare il valore della soluzione con il vertice $\gamma(6,2)$:
$$
\varphi(6,2)=-6-2=-8
$$
Anche il vertice $\gamma$ � una soluzione ottima del nostro problema. Da ci� possiamo desumere che l'intero spigolo $\OL{\beta\gamma}$ � composto da infinite soluzioni ottime. D'altronde, spostandoci lungo $\OL{\beta\gamma}$ avanzeremo in direzione perpendicolare al gradiente della funzione obiettivo e il valore della soluzione non pu� cambiare.
Si fa notare infine che la nostra funzione obiettivo iniziale �:
$$
z=-\varphi=8
$$

\subsection{Soluzione del problema primale}
La soluzione ottima consiste nel produrre $4.5$T di composto A e $3.5$T di composto B ottenendo un profitto pari a 8 volte quello di $1$T di composto A (o di composto B, equivalentemente).

\subsection{Costruzione del problema duale}
Riportiamo, per comodit�, il problema primale espresso in forma standard.
\begin{alignat*}{7}
&\min \varphi = \quad && -x_1 \quad\; && -x_2 \quad\;\; && \qquad\qquad && \qquad\qquad && \qquad\qquad && \\
&\;\st  &&+x_1			&&+x_2	 		&&+\pmb{x_3}	&&		 		&&					&&=8\\
&	 	&&+x_1			&&-x_2			&&				&& -\pmb{x_4}	&&					&&=1\\
&	 	&&+x_1			&&				&&				&&				&&+\pmb{x_5}		&&=6\\
&		&&\quad\; x_1,	&&\quad\; x_2,	&&\quad\; x_3,	&&\quad\; x_4,	&&\quad\; x_5		&&\geq 0
\end{alignat*}
Ricordiamo che le regole base per la creazione del problema duale (considereremo solo quelle in grassetto nel caso di problemi primali in forma standard):
\begin{itemize}
\item \textbf{Ad ogni vincolo corrisponde una variabile duale};
\item \textbf{Ad ogni vincolo di uguaglianza, la rispettiva variabile duale � una variabile libera};
\item Ad ogni vincolo di non minoranza corrisponde una variabile duale non negativa;
\item \textbf{Ad ogni variabile non negativa nel primale corrisponde un vincolo con relazione di non maggioranza nel duale};
\item Ad ogni variabile libera nel primale corrisponde un vincolo di uguaglianza nel duale.
\end{itemize}
In dettaglio, ridefiniamo in questo modo il generico problema primale in forma standard:
\begin{align*}
\min c'x& \\
Ax& = b \\
x& \geq 0
\end{align*}
Sia $\pi$ il vettore delle variabili duali, il problema duale � il seguente:
\begin{align*}
\max \pi'b& \\
\pi'A& \leq c' \\
\pi'&\gtreqless 0
\end{align*}
Ove, i vettori $x,\pi,b,c$ e la matrice $A$ sono:
\begin{align*}
x'&=
\begin{bmatrix}
x_1 & x_2 & x_3 & x_4 & x_5
\end{bmatrix} \\
\pi'&=
\begin{bmatrix}
\pi_1 & \pi_2 & \pi_3
\end{bmatrix} \\
b'&=
\begin{bmatrix}
8 & 1 & 6
\end{bmatrix} \\
c'&=
\begin{bmatrix}
-1 & -1 & 0 & 0 & 0
\end{bmatrix} \\
A&=
\begin{bmatrix}
1  & 1  & 1  & 0  & 0 \\
1  & -1 & 0  & -1 & 0 \\
1  & 0  & 0  & 0  & 1 \\
\end{bmatrix}
\end{align*}
Da ci�, il corrispondente problema duale con la sua funzione obiettivo $\xi$:
\begin{alignat*}{7}
&\max \xi = \quad && +8\pi_1 \quad\; && +\pi_2 \quad\;\; && +6\pi_3 \quad\;\; && \\
&\;\st  &&+\pi_1		&&+\pi_2 		&&+\pi_3		&& \leq -1\\
&	 	&&+\pi_1		&&-\pi_2 		&&				&& \leq -1\\
&	 	&&+\pi_1		&&				&&				&& \leq 0 \\
&	 	&&				&&-\pi_2		&&				&& \leq 0 \\
&	 	&&				&&				&&+\pi_3		&& \leq 0 \\
&		&&\quad\;\pi_1,	&&\quad\;\pi_2,	&&\quad\;\pi_3,	&& \gtreqless 0
\end{alignat*}
Per trovare la soluzione del problema duale non � necessario trasformarlo in forma standard e applicare il metodo del simplesso. Il tableau del problema primale sul quale abbiamo applicato il metodo del simplesso contiene tutte le informazioni per avere la soluzione del problema duale.
\subsubsection{Richiami (sempre molto blandi) di teoria}
Per ottenere dal tableau del problema primale la soluzione del problema duale, � sufficiente ricordare che il problema duale � ottenuto a partire dal \textbf{criterio di ottimalit�}.
Per questo motivo, il costo relativo nel tableau finale - corrispondente alla soluzione ottima - � cos� esprimibile:
$$
\OL{c_j}=c_j-z_j=c_j-\pi'A_j \quad \forall j
$$
Se consideriamo le colonne $A_j$ corrispondenti alla base iniziale $\mathcal{B}_0$ di partenza del primo tableau - ricordando che � una matrice identit� - possiamo ottenere:
$$
\OL{c_j}=c_j-\pi_j \quad \forall j:A_j\in\mathcal{B}_0
$$
Applicando un semplice passaggio algebrico:
$$
\pi_j=c_j-\OL{c_j}
$$
Ove $c_j$ � il costo iniziale nel primo tableau e $\OL{c_j}$ il costo relativo nel tableau finale.
Nel caso in cui abbiamo fatto uso di variabili artificiali e della fase 1 del metodo del simplesso, allora per tale variabile - il cui costo � $c_j=0$ - vale:
$$
\pi_j=-\OL{c_j}
$$
\textit{� importante ricordare che bisogna utilizzare il primo tableau con le variabili artificiali ma con il vettore dei costi originario in cui le variabili artificiali hanno costo nullo.}
\subsection{Soluzione del problema duale}
Riportiamo nuovamente i tableau iniziale e finale rispettivamente nelle tabella \vref{tab:tab49} e \vref{tab:tab410}.
\begin{table}[htbp]
\centering
\begin{tabular}{rrccccccc}
 	  & 			&$-\varphi$	& $x_1$ & $x_2$ & $x_3$ & $x_4$ & $x_5$	& $x^a$\\
$R_0$ & $\OL{c_j}$ 	& \Sc{0} 	& -1 	& -1 	& 0 	& 0 	& \Sc{0}& 0\\
\cline{3-9}
$R_1$ & $x_3$ 		& \Sc{8} 	& 1		& 1 	& 1 	& 0 	& \Sc{0}& 0 \\
$R_2$ & $x^a$ 		& \Sc{1} 	& 1		& -1	& 0 	& -1 	& \Sc{0}& 1 \\
$R_3$ & $x_5$ 		& \Sc{6} 	& 1 	& 0 	& 0 	& 0 	& \Sc{1}& 0 \\
\end{tabular}
\caption{Nuovo tableau con la variabile artificiale $x^a$.}
\label{tab:tab49}
\end{table}
\begin{table}[htbp]
\centering
{
	\newcommand{\sm}{$\frac{7}{2}$}
	\newcommand{\nm}{$\frac{9}{2}$}
	\newcommand{\um}{$\frac{1}{2}$}
	\newcommand{\tm}{$\frac{3}{2}$}
\begin{tabular}{rrccccccc}
 	  & 			&$-\varphi$	& $x_1$ & $x_2$ & $x_3$ & $x_4$ & $x_5$	& $x^a$\\
$R_0$ & $\OL{c_j}$ 	& \Sc{8} 	& 0 	& 0 	& 1 	& 0 	& \Sc{0}& 0\\
\cline{3-9}
$R_1$ & $x_2$ 		& \Sc{\sm}	& 0		& 1		& \um 	& \um 	& \Sc{0}& -\um \\
$R_2$ & $x_1$ 		& \Sc{\nm} 	& 1		& 0		& \um	& -\um 	& \Sc{0}& \um \\
$R_3$ & $x_5$ 		& \Sc{-\tm}	& 0 	& 0 	& -\um 	& \um 	& \Sc{1}& -\um \\
\end{tabular}
}
\caption{Tableau finale.}
\label{tab:tab410}
\end{table}
La base iniziale � $\mathcal{B}_1={A_3,A_6,A_5}$. Applicando delle semplici sottrazioni, ricaviamo la soluzione del problema duale:
\begin{align*}
\pi_1&=c_3-\OL{c_3}=-1 \\
\pi_2&=c_6-\OL{c_6}=0 \\
\pi_3&=c_5-\OL{c_5}=0
\end{align*}
Perci�, la soluzione del problema duale � il vettore;
$$
\pi'=
\begin{bmatrix}
-1 & 0 & 0
\end{bmatrix}
$$
Per verificare la correttezza dei calcoli, applichiamo la soluzione alla funzione obiettivo del problema duale:
$$
\xi(-1,0,0)=8(-1)+0+6(0)=-8
$$
Il risultato �, come atteso, lo stesso del problema primale.

\chapter{15/04/2014}

In questo capitolo, riguardante la terza esercitazione, verranno svolti alcuni dei punti mancanti dell'esercitazione precedente. Per essere precisi, tutti quelli riguardanti il \textbf{vincolo di interezza} sulle variabili, cio� i problemi ILP. Non saranno riportati, ovviamente, gli interi esercizi ma solo i tableau finali con i quali partire per eseguire il simplesso duale. Verranno comunque creati dei riferimenti ipertestuali negli esercizi precedenti che rimanderanno alle soluzioni che di seguito apporr�.

\section{Esercizio 4}

In questa sezione riprenderemo da dove eravamo rimasti in \vref{sec:es4ILP}, risolvendo il problema ILP prima con il metodo dei \textbf{tagli di Gomory} e poi con il metodo \textbf{branch and bound} (il primo in realt� non � richiesto dalla traccia, ma lo proveremo ugualmente come utile esercitazione).
Prima di ci�, � bene ricordare alcuni concetti teorici.

\subsection{Richiami (blandi, sempre blandi) di teoria}
\subsubsection{Tagli di Gomory}
Ricordiamo dato un problema ILP con \textbf{soluzione del rilassamento continuo $x^*$}, \textbf{eseguire un taglio} significa applicare un \textbf{vincolo} tale che:
\begin{enumerate}
\item \textbf{elimini} una parte della regione ammissibile \textbf{contente $x^*$};
\item \textbf{non elimini} nessuna soluzione intera ammissibile.
\end{enumerate}
Definendo:
\begin{align*}
\mathcal{B} & \qquad\textbf{base ottima} \\
f_{ij}=y_{ij}-\lfloor y_{ij} \rfloor & \qquad\textbf{parte frazionaria di $y_{ij}$}
\end{align*}
Un \textbf{taglio di Gomory} � tale che:
$$
\sum_{A_j \notin \mathcal{B}} f_{ij}x_j \geq f_{i0}
$$
Ove la riga $i$ � scelta arbitrariamente tra quelle per cui $y_{i0}\notin\mathbb{N}$.
Pu� essere dimostrato (ma non lo faremo qui) che questo tipo di taglio rispetta i due punti citati in precedenza.
\subsubsection{Branch and bound (da rivedere)}
Come dal nome del metodo, questo metodo si compone di due parti: \textbf{branching} e \textbf{bounding}.
\paragraph{Branching}
L'operazione di \textbf{branching} consiste nell'imporre due \textbf{vincoli mutualmente esclusivi ed esaustivi} su una variabile $x_j$, data $x_j^*$ la componente $j$-esima della soluzione del rilassamento continuo:
$$
x_j \leq \lfloor x_j^* \rfloor \qquad\veebar\qquad x_j \geq \lceil x_j^* \rceil
$$
Unendo al problema originario alternativamente questi due vincoli, si ottengono due nuovi sottoproblemi. La soluzione migliore tra i due sar� anche la migliore in assoluto. Per risolvere i due nuovi sottoproblemi si pu� applicare ricorsivamente un nuovo branching.
\paragraph{Bounding}
Ogni volta che si ottiene un nuovo sottoproblema da quello originario, si risolve nuovamente il rilassamento continuo.
Il rilassamento continuo di un problema costituisce un \textbf{lower bound}, poich� nessuna soluzione pu� essere migliore di quella offerta dal rilassamento continuo. In particolare, si pu� considerare come lower bound anche la sua \textbf{parte intera}.
Nel momento in cui uno dei sottoproblemi offre una soluzione intera uguale a quella del rilassamento continuo, si pu� evitare di risolvere \textbf{tutti i sottoproblemi non ancora risolti il cui lower bound � maggiore o uguale alla soluzione appena ottenuto} in quanto nella migliore delle ipotesi potremmo trovare solo una soluzione uguale a quella appena trovata, ma non una migliore.
\paragraph{Rappresentazione}
Il metodo migliore per rappresentare la soluzione tramite branch and bound � un albero binario.

\begin{figure}[h!]
\centering
\begin{tikzpicture}[edge from parent/.style={draw=black,-latex},
					level/.style={sibling distance=50mm, level distance=20mm},
					cross/.style={path picture={ 
 						\draw[black]
						(path picture bounding box.south east) -- (path picture bounding box.north west) (path picture bounding box.south west) -- (path picture bounding box.north east);}}
					]
\node [circle, draw, label=east:{$L_B=-9$}] (z) {$P_0$}
	child {node [circle, draw, label=east:{$L_B=-8$}] (a) {$P_1$}
			child {node [circle, draw, label=east:{$L_B=-8$}, label=south:{$-z=-8$}] (b) {$P_2$}
				   edge from parent node[above left] {$x_2\leq 3$}
				   }
			edge from parent node[left] {$x_1\leq 1$}
		   } 
	child {node [circle, draw, cross] (b) {$P_3$}
			edge from parent node[above right, label=east:{$L_B=-8$}] {$x_1\geq 2$}
			}
	;
\end{tikzpicture}
\end{figure}
Nell'esempio, il problema $P_2$ ha la soluzione del rilassamento continua uguale al lower bound del problema padre. Questo rende inutile esplorare l'altro figlio del problema $P_1$, in quanto non potremmo avere alcuna soluzione pi� bassa di $-8$, come indicato dal suo lower bound.
Poich� il lower bound del problema $P_0$ � migliore della soluzione attualmente disponibile, ha senso esplorare il suo altro figlio. Nel nostro esempio, per�, il figlio $P_3$ ha un lower bound uguale a $-8$ e non c'� pi� possibilit� di trovare una soluzione migliore di quella trovata in precedenza. Si dice, in questo caso, che il nodo $P_2$ \textbf{uccide} il nodo $P_3$ (sul quale abbiamo all'uopo apposto una croce).
\textit{Noto che � difficile spiegare in maniera generale il funzionamento del metodo branch and bound. Si spera perci� che risulti pi� chiaro applicandolo praticamente negli esercizi.}

\subsection{ILP - Tagli di Gomory}

Riportiamo in tabella \vref{tab:tab411} il tableau finale della soluzione del problema LP.
\begin{table}[htbp]
\centering
{
	\newcommand{\sm}{$\frac{7}{2}$}
	\newcommand{\nm}{$\frac{9}{2}$}
	\newcommand{\um}{$\frac{1}{2}$}
	\newcommand{\tm}{$\frac{3}{2}$}
\begin{tabular}{rrccccccc}
 	  & 			&$-\varphi$	& $x_1$ & $x_2$ & $x_3$ & $x_4$ & $x_5$	& $x^a$\\
$R_0$ & $\OL{c_j}$ 	& \Sc{8} 	& 0 	& 0 	& 1 	& 0 	& \Sc{0}& 0\\
\cline{3-9}
$R_1$ & $x_2$ 		& \Sc{\sm}	& 0		& 1		& \um 	& \um 	& \Sc{0}& -\um \\
$R_2$ & $x_1$ 		& \Sc{\nm} 	& 1		& 0		& \um	& -\um 	& \Sc{0}& \um \\
$R_3$ & $x_5$ 		& \Sc{\tm}	& 0 	& 0 	& -\um 	& \um 	& \Sc{1}& -\um \\
\end{tabular}
}
\caption{Tableau finale del problema LP. Rappresenta il rilassamento continuo del problema ILP.}
\label{tab:tab411}
\end{table}
Le colonne $R_1$,$R_2$ ed $R_3$ sono valide candidate per  applicare il taglio di Gomory. Scegliamo la riga $R_1$:
$$
\frac{1}{2}x_3+\frac{1}{2}x_4\geq\frac{1}{2}
$$
Portiamo in forma standard il vincolo moltiplicando per $-1$ e aggiungendo una variabile slack $s$:
$$
-\frac{1}{2}x_3-\frac{1}{2}x_4+s=\frac{1}{2}
$$
Il vincolo � abbastanza agevole da inserire nel tableau come in tabella \vref{tab:tab412} (� stata eliminata la variabile artificiale, ormai inutile).
\begin{table}[htbp]
\centering
{
	\newcommand{\sm}{$\frac{7}{2}$}
	\newcommand{\nm}{$\frac{9}{2}$}
	\newcommand{\um}{$\frac{1}{2}$}
	\newcommand{\mum}{$-\frac{1}{2}$}
	\newcommand{\tm}{$\frac{3}{2}$}
\begin{tabular}{rrccccccc}
 	  & 			&$-\varphi$	& $x_1$ & $x_2$ & $x_3$ & $x_4$ & $x_5$	& $s$\\
$R_0$ & $\OL{c_j}$ 	& \Sc{8} 	& 0 	& 0 	& 1 	& 0 	& \Sc{0}& 0\\
\cline{3-9}
$R_1$ & $x_2$ 		& \Sc{\sm}	& 0		& 1		& \um 	& \um 	& \Sc{0}& 0 \\
$R_2$ & $x_1$ 		& \Sc{\nm} 	& 1		& 0		& \um	& \mum 	& \Sc{0}& 0 \\
$R_3$ & $x_5$ 		& \Sc{\tm}	& 0 	& 0 	& \mum 	& \um 	& \Sc{1}& 0 \\
\cline{3-8}
$R_4$ & $s$			& \Sc{\mum}	& 0		& 0 	& \mum	&\C{\mum}& 0	& 1
\end{tabular}
}
\caption{Applicazione del primo taglio di Gomory.}
\label{tab:tab412}
\end{table}
� evidente che nel tableau � ancora presente la sottomatrice identit� e che la soluzione rispetta il \textit{criterio di ottimalit�} essendo tutti i $\OL{c_j}$ non negativi. L'unico intoppo � rappresentato dal valore della variabile $s$, che � negativo.
In altri termini, siamo in una soluzione ottima ma non ammissibile. Lo strumento che fa al caso nostro ora � il \textbf{simplesso duale}, che ci porter� da una base ottima ad una nuova base sempre ottima ma pi� vicina all'ammissibilit�.
Il simplesso duale opera come il simplesso primale ma invertendo righe e colonne. Perci�, dovremo scegliere tra tutti gli elementi della \textit{riga} $i$-esima con $y_{i0}<0$ quello su cui fare pivot. A tal scopo, ridefiniamo la grandezza $\vartheta$ in tal modo:
$$
\vartheta=\max_{j>0:y_{ij}<0}\left(\frac{y_{0j}}{y_{ij}}\right)=\frac{y_{0s}}{y_{is}}
$$
L'elemento $y_{is}$ sar� quello di pivoting.
Nel nostro caso:
$$
\vartheta=\max\left(\frac{1}{-\frac{1}{2}},\frac{0}{-\frac{1}{2}}\right)=\frac{0}{-\frac{1}{2}}=\frac{y_{04}}{\pmb{y_{44}}}
$$
Faremo pivoting sull'elemento $y_{44}$, cerchiato in tabella \vref{tab:tab412}. Le operazioni da attuare sono le consuete operazioni elementari di riga:
\begin{align*}
R_1&\leftarrow R_1+R_4 \\
R_2&\leftarrow R_1-R_4 \\
R_3&\leftarrow R_3+R_4 \\
R_4&\leftarrow -2R_4
\end{align*}
Ci� che ne risulta � il tableau in tabella \vref{tab:tab413}.
\begin{table}
\centering
{
	\newcommand{\sm}{$\frac{7}{2}$}
	\newcommand{\nm}{$\frac{9}{2}$}
	\newcommand{\um}{$\frac{1}{2}$}
	\newcommand{\mum}{$-\frac{1}{2}$}
	\newcommand{\tm}{$\frac{3}{2}$}
\begin{tabular}{rrccccccc}
 	  & 			&$-\varphi$	& $x_1$ & $x_2$ & $x_3$ & $x_4$ & $x_5$	& $s$\\
$R_0$ & $\OL{c_j}$ 	& \Sc{8} 	& 0 	& 0 	& 1 	& 0 	& \Sc{0}& 0\\
\cline{3-9}
$R_1$ & $x_2$ 		& \Sc{3}	& 0		& 1		& 0 	& 0 	& \Sc{0}& 1 \\
$R_2$ & $x_1$ 		& \Sc{5} 	& 1		& 0		& 1		& 0 	& \Sc{0}& -1 \\
$R_3$ & $x_5$ 		& \Sc{1}	& 0 	& 0 	& -1 	& 0 	& \Sc{1}& 1 \\
\cline{3-8}
$R_4$ & $x_4$		& \Sc{1}	& 0		& 0 	& 1		& 1		& 0		& -2
\end{tabular}
}
\caption{Dopo il taglio di Gomory siamo in una soluzione intera nel vertice $\varepsilon(5,3)$.}
\label{tab:tab413}
\end{table}
La nuova base e la nuova soluzione sono:
\begin{align*}
\mathcal{B}&={A_2,A_1,A_5,A_4} \\
x&=(5,3,0,1,1)
\end{align*}
La soluzione � intera ed il nostro problema di ILP � risolto nel nuovo vertice $\varepsilon(5,3)$. Il valore della soluzione �:
$$
z(\varepsilon)=-\varphi=8
$$
\subsubsection{Rappresentazione grafica}
Per quanto riguarda i tagli di Gomory, la rappresentazione grafica non � agevole e immediata e se ne dar� una rappresentazione solo alla fine.
Il taglio, in forma di disequazione, � il seguente:
$$
\frac{1}{2}x_3+\frac{1}{2}x_4\geq\frac{1}{2}
$$
Ovviamente non � rappresentabile, in questa forma algebrica, sul piano cartesiano. Le variabili $x_3$ e $x_4$ sono rispettivamente una variabile slack e una variabile surplus, perci� possiamo fare in modo di sostituirle con le variabili reali $x_1$ e $x_2$ operando algebricamente sui vincoli in forma standard che le introducono:
$$
\begin{cases}
x_1 + x_2 + x_3 = 8 \\
x_1 - x_2 - x_4 = 1 
\end{cases}
\Rightarrow
\begin{cases}
x_3 = 8 - x_1 - x_2 \\
x_4 = -1 + x_1 - x_2
\end{cases}
$$
Sostituiamo le due variabili nella disequazione:
$$
\frac{1}{2}(8-x_1-x_2)+\frac{1}{2}(-1+x_1-x_2)\geq\frac{1}{2} \Rightarrow x_2 \leq 3
$$
Cos� espresso il vincolo � ora facilmente rappresentabile graficamente come in figura \vref{fig:graph42}.
\begin{figure}[htbp]
\centering
\begin{tikzpicture}
\begin{axis}
[axis lines=middle, axis equal, enlargelimits, xlabel=$x_1$, ylabel=$x_2$,
 every axis x label/.style={
    at={(ticklabel* cs:1.01)},
    anchor=west,
 },
 every axis y label/.style={
    at={(ticklabel* cs:1.01)},
    anchor=south,
 },]
    \path[name path=AX] 
        (axis cs:\pgfkeysvalueof{/pgfplots/xmin},0)--
        (axis cs:\pgfkeysvalueof{/pgfplots/xmax},0);
    \path[name path=AY] 
        (axis cs:0,\pgfkeysvalueof{/pgfplots/ymin})--
        (axis cs:0,\pgfkeysvalueof{/pgfplots/ymax});
    \path[name path=UP]
    	(axis cs:\pgfkeysvalueof{/pgfplots/xmin},\pgfkeysvalueof{/pgfplots/ymax})--
    	(axis cs:\pgfkeysvalueof{/pgfplots/xmax},\pgfkeysvalueof{/pgfplots/ymax});
\dotgrid{0}{8}{0}{4}{black!50}
\addplot
[domain=0:8, samples=10, thick, blue, name path=xy8] {-x+8};
\addplot
[domain=0:8, samples=10, thick, red, name path=xy1] {x-1};
\addplot
[domain=0:8, samples = 10, thick, purple, name path=x6] (6,x);
\addplot
[domain=0:8, samples=10, very thick, green, name path=gomory]
{3} node [pos=0.8, anchor=south, pin={75:{\color{green}$x_2\leq 3$}}, inner sep= 0pt] {};
\addplot[thick, fill=yellow, fill opacity=0.5] fill between [of=xy1 and AX, soft clip={domain=1:6}];
\addplot[white] fill between [of=xy8 and UP];
\addplot[white] fill between [of=x6 and AX];
\addplot[white] fill between [of=gomory and UP];
%\addplot[pattern=north east lines, pattern color=blue!10] fill between [of=xy8 and AX];
\ints{AX}{xy1}{$\alpha$}{alp};
\ints{xy1}{xy8}{$\beta$}{bet};
\intw{xy8}{x6}{$\gamma$}{gam};
\intnw{x6}{AX}{$\delta$}{del};
\intne{gomory}{xy8}{$\varepsilon$}{eps};
\node at (axis cs:4.5,1.5) {$P'$};
    \path[name path=GOM] 
        (axis cs:\pgfkeysvalueof{/pgfplots/xmin}, 2.5)--
        (axis cs:\pgfkeysvalueof{/pgfplots/xmax}, 2.5);
\addplot[pattern=north east lines, pattern color=green!30] fill between [of=gomory and GOM, soft clip={domain=0:8}];
\addplot[-latex, thick] coordinates
           {(0,0) (1/1.414,1/1.414)} node [pos=.3, anchor=south, label={45:{\small $\nabla z$}}] {};
\end{axis}
\end{tikzpicture}
\caption{Rappresentazione cartesiana del problema di programmazione lineare}
\label{fig:graph42}
\end{figure}
In giallo � rappresentato il nuovo politopo $P'$ risultato del vecchio politopo $P$ e del taglio di Gomory applicato.

\subsection{ILP - Branch and bound}
Il problema di LP iniziale verr� denominato $P_0$. La soluzione del rilassamento continuo $x^*=\left(\frac{9}{2},\frac{7}{2}\right)$ � $-z^*=-8$, il che imposter� il nostro lower bound a $L_B=\lceil -8 \rceil = -8$.
\begin{figure}[h!]
\centering
\begin{tikzpicture}[edge from parent/.style={draw=black,-latex},
					level/.style={sibling distance=50mm, level distance=20mm},
					cross/.style={path picture={ 
 						\draw[black]
						(path picture bounding box.south east) -- (path picture bounding box.north west) (path picture bounding box.south west) -- (path picture bounding box.north east);}}
					]
\node [circle, draw, label=east:{$L_B=-8$}] (z) {$P_0$}
	;
\end{tikzpicture}
\end{figure}

Iniziando dalla variabile $x_1$ possiamo prendere in considerazione due diversi sottoproblemi derivati dall'aggiunta di uno dei due seguenti vincoli:
\begin{align*}
x_1&\geq\lceil x^*_1 \rceil \\
x_1&\leq\lfloor x^*_1 \rfloor
\end{align*}
Per convincerci che considerando questi due vincoli non si esclude nessuna soluzione e che sono mutuamente esclusivi, si faccia riferimento al grafico \vref{fig:graph43} e osservando che le due aree colorate in ciano e in verde non escludono nessuno dei punti interi.
\begin{figure}[htbp]
\centering
\begin{tikzpicture}
\begin{axis}
[axis lines=middle, axis equal, enlargelimits, xlabel=$x_1$, ylabel=$x_2$,
 every axis x label/.style={
    at={(ticklabel* cs:1.01)},
    anchor=west,
 },
 every axis y label/.style={
    at={(ticklabel* cs:1.01)},
    anchor=south,
 },]
    \path[name path=AX] 
        (axis cs:\pgfkeysvalueof{/pgfplots/xmin},0)--
        (axis cs:\pgfkeysvalueof{/pgfplots/xmax},0);
    \path[name path=AY] 
        (axis cs:0,\pgfkeysvalueof{/pgfplots/ymin})--
        (axis cs:0,\pgfkeysvalueof{/pgfplots/ymax});
    \path[name path=UP]
    	(axis cs:\pgfkeysvalueof{/pgfplots/xmin},\pgfkeysvalueof{/pgfplots/ymax})--
    	(axis cs:\pgfkeysvalueof{/pgfplots/xmax},\pgfkeysvalueof{/pgfplots/ymax});
\dotgrid{0}{8}{0}{4}{black!50}
\addplot
[domain=4:8, samples=10, thick, blue, name path=xy8] {-x+8};
\addplot
[domain=1:5, samples=10, thick, red, name path=xy1] {x-1};
\addplot
[domain=0:3, samples = 10, thick, purple, name path=x6] (6,x);
\addplot
[domain=0:4, samples=10, thick, cyan, name path=bb1]
(4,x) node [pos=0, anchor=south, pin={-135:{\color{cyan}$x_1\leq 4$}}, inner sep= 0pt] {};
    \path[name path=bb1i] (axis cs:3.75, 0)--(axis cs:3.75, 4);
\addplot
[domain=0:4, samples=10, thick, green, name path=bb2]
(5,x) node [pos=0, anchor=south, pin={-75:{\color{green}$x_2\geq 5$}}, inner sep= 0pt] {};
    \path[name path=bb2i] (axis cs:5.25, 0)--(axis cs:5.25, 4);
\addplot[thick, fill=cyan, fill opacity=0.2] fill between [of=bb1 and xy1];
\addplot[white] fill between [of=xy1 and UP];
\addplot[thick, fill=green, fill opacity=0.2] fill between [of=bb2 and x6];
\addplot[white] fill between [of=xy8 and UP];
%\addplot[pattern=north east lines, pattern color=blue!10] fill between [of=xy8 and AX];
\ints{AX}{xy1}{$\alpha$}{alp};
\inte{xy1}{xy8}{$\beta\equiv x^*$}{bet};
\intw{xy8}{x6}{$\gamma$}{gam};
\intnw{x6}{AX}{$\delta$}{del};
\inte{gomory}{xy8}{$\varepsilon$}{eps};
\node at (axis cs:3.5,1.5) {$P_2$};
\node at (axis cs:5.5,1.5) {$P_1$};
\addplot[pattern=north east lines, pattern color=cyan] fill between [of=bb1 and bb1i];
\addplot[pattern=north east lines, pattern color=green] fill between [of=bb2 and bb2i];
\addplot[-latex, thick] coordinates
           {(0,0) (1/1.414,1/1.414)} node [pos=.3, anchor=south, label={45:{\small $\nabla z$}}] {};
\end{axis}
\end{tikzpicture}
\caption{Rappresentazione cartesiana dei due sottoproblemi di $P_0$.}
\label{fig:graph43}
\end{figure}
Esploriamo ora il figlio $P_1$ del problema $P_0$, corrispondente al vincolo $x_1\geq\lceil x^*_1 \rceil = 5$.
\begin{figure}[h!]
\centering
\begin{tikzpicture}[edge from parent/.style={draw=black,-latex},
					level/.style={sibling distance=50mm, level distance=20mm},
					cross/.style={path picture={ 
 						\draw[black]
						(path picture bounding box.south east) -- (path picture bounding box.north west) (path picture bounding box.south west) -- (path picture bounding box.north east);}}
					]
\node [circle, draw, label=east:{$L_B=-8$}] (z) {$P_0$}
	child {node [circle, draw] (a) {$P_1$}
			edge from parent node[left] {$x_1\geq 5$}
		   }
	;
\end{tikzpicture}
\end{figure}

In \textit{forma standard} il vincolo pu� essere cos� espresso:
$$
-x_1 + s = -5
$$
In questa formula risulta molto agevole l'introduzione nel tableau iniziale di $P_0$ (tabella \vref{tab:tab411}) come in tabella \vref{tab:tab414}.
\begin{table}[htbp]
\centering
{
	\newcommand{\sm}{$\frac{7}{2}$}
	\newcommand{\nm}{$\frac{9}{2}$}
	\newcommand{\um}{$\frac{1}{2}$}
	\newcommand{\mum}{$-\frac{1}{2}$}
	\newcommand{\tm}{$\frac{3}{2}$}
\begin{tabular}{rrccccccc}
 	  & 			&$-\varphi$	& $x_1$ & $x_2$ & $x_3$ & $x_4$ & $x_5$	& $s$\\
$R_0$ & $\OL{c_j}$ 	& \Sc{8} 	& 0 	& 0 	& 1 	& 0 	& \Sc{0}& 0\\
\cline{3-9}
$R_1$ & $x_2$ 		& \Sc{\sm}	& 0		& 1		& \um 	& \um 	& \Sc{0}& 0 \\
$R_2$ & $x_1$ 		& \Sc{\nm} 	& 1		& 0		& \um	& \mum 	& \Sc{0}& 0 \\
$R_3$ & $x_5$ 		& \Sc{\tm}	& 0 	& 0 	& \mum 	& \um 	& \Sc{1}& 0 \\
\cline{3-8}
$R_4$ & $s$			& \Sc{-5}	& -1	& 0 	& 0		& 0 	& 0		& 1
\end{tabular}
}
\caption{Sottoproblema $P_1$.}
\label{tab:tab414}
\end{table}
Poich� $y_{41}=-5$ rovina la nostra matrice identit� in base, operiamo la seguente operazione elementare di riga:
$$
R_4\leftarrow R_4 + R_2
$$
Si giunge quindi al tableau in tabella \vref{tab:tab415}.
\begin{table}[htbp]
\centering
{
	\newcommand{\sm}{$\frac{7}{2}$}
	\newcommand{\nm}{$\frac{9}{2}$}
	\newcommand{\um}{$\frac{1}{2}$}
	\newcommand{\mum}{$-\frac{1}{2}$}
	\newcommand{\tm}{$\frac{3}{2}$}
\begin{tabular}{rrccccccc}
 	  & 			&$-\varphi$	& $x_1$ & $x_2$ & $x_3$ & $x_4$ & $x_5$	& $s$\\
$R_0$ & $\OL{c_j}$ 	& \Sc{8} 	& 0 	& 0 	& 1 	& 0 	& \Sc{0}& 0\\
\cline{3-9}
$R_1$ & $x_2$ 		& \Sc{\sm}	& 0		& 1		& \um 	& \um 	& \Sc{0}& 0 \\
$R_2$ & $x_1$ 		& \Sc{\nm} 	& 1		& 0		& \um	& \mum 	& \Sc{0}& 0 \\
$R_3$ & $x_5$ 		& \Sc{\tm}	& 0 	& 0 	& \mum 	& \um 	& \Sc{1}& 0 \\
\cline{3-8}
$R_4$ & $s$			& \Sc{\mum}	& 0		& 0 	& \um	&\C{\mum}& 0	& 1
\end{tabular}
}
\caption{Sottoproblema $P_1$, secondo tableau.}
\label{tab:tab415}
\end{table}

Si pu� ora procedere con il simplesso duale. L'unico elemento di $R_4$ negativo sul quale � possibile fare pivoting � $y_{44}$. Le operazioni elementari di riga sono, nell'ordine:
\begin{align*}
R_1&\leftarrow R_1 + R_4 \\
R_2&\leftarrow R_2 - R_4 \\
R_3&\leftarrow R_3 + R_4 \\
R_4&\leftarrow -2R_4
\end{align*}
Ne risulta il tableau in tabella \vref{tab:tab416}.
\begin{table}
\centering
{
	\newcommand{\sm}{$\frac{7}{2}$}
	\newcommand{\nm}{$\frac{9}{2}$}
	\newcommand{\um}{$\frac{1}{2}$}
	\newcommand{\mum}{$-\frac{1}{2}$}
	\newcommand{\tm}{$\frac{3}{2}$}
\begin{tabular}{rrccccccc}
 	  & 			&$-\varphi$	& $x_1$ & $x_2$ & $x_3$ & $x_4$ & $x_5$	& $s$\\
$R_0$ & $\OL{c_j}$ 	& \Sc{8} 	& 0 	& 0 	& 1 	& 0 	& \Sc{0}& 0\\
\cline{3-9}
$R_1$ & $x_2$ 		& \Sc{3}	& 0		& 1		& 0 	& 0 	& \Sc{0}& 1 \\
$R_2$ & $x_1$ 		& \Sc{5} 	& 1		& 0		& 0		& 0 	& \Sc{0}& -1 \\
$R_3$ & $x_5$ 		& \Sc{1}	& 0 	& 0 	& 0 	& 0 	& \Sc{1}& 1 \\
\cline{3-8}
$R_4$ & $x_4$		& \Sc{1}	& 0		& 0 	& -1		& 1		& 0		& -2
\end{tabular}
}
\caption{Siamo in una soluzione intera nel vertice $\varepsilon(5,3)$.}
\label{tab:tab416}
\end{table}
La nuova base e la nuova soluzione sono:
\begin{align*}
\mathcal{B}&={A_2,A_1,A_5,A_4} \\
x&=(5,3,0,1,1)
\end{align*}
La soluzione � intera ed il ramo attuale dell'albero � giunto ad una foglia.
\begin{figure}[h!]
\centering
\begin{tikzpicture}[edge from parent/.style={draw=black,-latex},
					level/.style={sibling distance=50mm, level distance=20mm},
					cross/.style={path picture={ 
 						\draw[black]
						(path picture bounding box.south east) -- (path picture bounding box.north west) (path picture bounding box.south west) -- (path picture bounding box.north east);}}
					]
\node [circle, draw, label=east:{$L_B=-8$}] (z) {$P_0$}
	child {node [circle, draw, label=east:{$L_B=-8$}, label=south:{$-z=-8$}] (a) {$P_1$}
			edge from parent node[left] {$x_1\geq 5$}
		   }
	;
\end{tikzpicture}
\end{figure}

Siamo fortunati perch� l'attuale soluzione � uguale al lower bound del nodo \textit{root}, il che significa che non serve cercare altre soluzioni che potrebbero essere al pi� uguali a quella trovata. Ne conviene infine che il vertice $\varepsilon(5,3)$ � una soluzione ottima del problema ILP. Il valore della soluzione �:
$$
z(\varepsilon)=-\varphi=8
$$

\end{document}
